%----------------------------------------------------------------------------------------
%	PACKAGES AND OTHER DOCUMENT CONFIGURATIONS
%----------------------------------------------------------------------------------------

\documentclass[letterpaper]{article}

\usepackage[utf8]{inputenc} % Required for inputting international characters
% \usepackage[T1]{fontenc} % Output font encoding for international characters
\usepackage[margin=1in]{geometry}
% \usepackage{mathpazo} % Palatino font
\usepackage[spanish]{babel}\decimalpoint % Soporte para español
\usepackage{graphicx} % Paquete para imágenes
\usepackage{fancyhdr} % Paquete para customización de las páginas
\usepackage{lastpage} % Saber última página
\usepackage[hidelinks]{hyperref} % Links para títulos
\usepackage{amsmath}
\usepackage{amssymb}
\usepackage{amsfonts}
\usepackage{xcolor}
\usepackage{empheq}
\usepackage{fancybox}
\usepackage[labelfont=bf]{caption}
\usepackage{array}
\usepackage{float}
\usepackage[most]{tcolorbox}
\usepackage{bookmark}
\usepackage{wrapfig}
\usepackage{multicol}
\usepackage{enumitem}

\newtcolorbox{greenbox}{
enhanced,
boxrule=0pt,frame hidden,
borderline west={4pt}{0pt}{green!75!black},
colback=green!10!white,
sharp corners
}

\newtcolorbox{redbox}{
enhanced,
boxrule=0pt,frame hidden,
borderline west={4pt}{0pt}{red!75!black},
colback=red!10!white,
sharp corners
}

\newtcolorbox{bluebox}{
enhanced,
boxrule=0pt,frame hidden,
borderline west={4pt}{0pt}{blue!75!black},
colback=blue!10!white,
sharp corners
}

{}\newtcbtheorem[auto counter,number within=section]{ejem}%
  {Ejemplo}{
      fonttitle=\bfseries\upshape, 
    %   fontupper=\slshape,
      arc=0mm, 
      colback=gray!15!white,
      breakable, 
      enhanced jigsaw,
      colframe=gray!75!black}
      {ejemplo}

\newtcbtheorem[auto counter]{theo}%
{Teorema}{fonttitle=\bfseries\upshape, fontupper=\slshape,
    arc=0mm, colback=white,colframe=black!85!white}{theorem}


\pagestyle{fancy} % Definir estilo de páginas
\fancyhf{}
% \lhead{\textbf{\footnotesize\rightmark}}
\rhead{\textbf{\footnotesize\leftmark}}
\lfoot{\textbf{\footnotesize IIC2223}}
\cfoot{\textbf{\footnotesize Teoría de Autómatas y Lenguajes Formales}}
\rfoot{\textbf{\footnotesize Página: \thepage\ de \pageref{LastPage}}}
\setlength{\shadowsize}{2pt}
\setlength{\fboxsep}{5pt}
\setlength\parindent{0pt}
\setlength{\headheight}{13.07225pt}
\bookmarksetup{numbered}
\setlist[itemize]{leftmargin=*}
\setlist[enumerate]{leftmargin=*}

% Renombre de comandos
\addto\captionsspanish{% Replace "english" with the language you use
  \renewcommand{\contentsname}
    {\LARGE Índice}
}
\addto\captionsspanish{% Replace "english" with the language you use
  \renewcommand{\tablename}
    {Tabla}
}
\renewcommand\arraystretch{1.5}
\renewcommand{\labelitemi}{{\raisebox{.3\height}{\scalebox{0.6}{$\blacklozenge$}}}}
\renewcommand{\headrulewidth}{0.4pt}
\renewcommand{\footrulewidth}{0.4pt}

% Nuevos comandos
\newcommand{\p}{\vspace{1em}}
\empheqset{marginbox=\psframebox}
\newcommand{\alignformula}[1]{
    \begin{empheq}[box=\shadowbox*]{align*}
        #1
    \end{empheq}
}
\newcommand{\fig}[3]{
    \begin{figure}[H]
        \centering
        \includegraphics[scale=#2]{#1}
        \caption{#3}
        \end{figure}
}
\newcommand{\img}[2]{
    \begin{figure}[H]
        \centering
        \includegraphics[scale=#2]{#1}
        \end{figure}
}
\newcommand{\enlace}[3]{\textcolor{#3}{\textbf{\href{#1}{#2}}}}
\newcommand{\ejemplo}[3]{\begin{ejem}{#1}{#2}#3\end{ejem}}
\newcommand{\teorema}[3]{\vspace{0.5em}\begin{theo}{#1}{#2}#3\end{theo}\vspace{0.5em}}
\newcommand{\li}[2]{\displaystyle{\lim_{#1 \to #2}}}
\newcommand{\ca}[1]{\mathcal{#1}}

\usepackage{etoolbox}
\apptocmd{\lim}{\limits}{}{}

\begin{document}

%----------------------------------------------------------------------------------------
%	TITLE PAGE
%----------------------------------------------------------------------------------------

\begin{titlepage} % Suppresses displaying the page number on the title page and the subsequent page counts as page 1
    \newcommand{\HRule}{\rule{\linewidth}{0.5mm}} % Defines a new command for horizontal lines, change thickness here

    \center % Centre everything on the page

    %------------------------------------------------
    %	Headings
    %------------------------------------------------

    \textsc{\LARGE Pontificia Universidad Católica de Chile}\\[1.5cm] % Main heading such as the name of your university/college

    \textsc{\Large Departamento de Ciencia de la Computación}\\[0.5cm] % Major heading such as course name

    \textsc{\large Apunte IIC2223}\\[0.5cm] % Minor heading such as course title

    %------------------------------------------------
    %	Title
    %------------------------------------------------

    \HRule\\[0.5cm]

    {\huge\bfseries Teoría de Autómatas y Lenguajes Formales}\\[0.4cm] % Title of your document

    \HRule\\[1.5cm]

    %------------------------------------------------
    %	Author(s)
    %------------------------------------------------

    \begin{minipage}{0.4\textwidth}
        \begin{flushleft}
            \large
            \textit{Autor}\\
            Cristóbal \textsc{Rojas} % Your name
        \end{flushleft}
    \end{minipage}
    ~
    \begin{minipage}{0.4\textwidth}
        \begin{flushright}
            \large
            \textit{En base a apuntes de}\\
            Prof. Cristian \textsc{Riveros} % Supervisor's name
        \end{flushright}
    \end{minipage}

    % If you don't want a supervisor, uncomment the two lines below and comment the code above
    % {\large\textit{Autor}}\\
    % Cristóbal \textsc{Rojas} % Your name

    %------------------------------------------------
    %	Date
    %------------------------------------------------

    \vfill\vfill\vfill % Position the date 3/4 down the remaining page

    {\large\today} % Date, change the \today to a set date if you want to be precise

    %------------------------------------------------
    %	Logo
    %------------------------------------------------

    \vfill\vfill
    \includegraphics[width=0.3\textwidth]{img/logo.png}\\[1cm] % Include a department/university logo - this will require the graphicx package

    %----------------------------------------------------------------------------------------

    \vfill % Push the date up 1/4 of the remaining page

\end{titlepage}

% \thispagestyle{empty}
% \begin{redbox}
%     \textbf{Advertencia:} Este no es un documento oficial del curso Cálculo I - MAT1610, el apunte puede presentar fallas y por ende generar error en sus cálculos. Debido a esto, el autor \textbf{no} se hace responsable de cualquier respuesta errónea que pueda dar en alguna prueba, por lo que es su obligación revisar si los contenidos mostrados aquí coinciden con el material oficial del curso. De cualquier manera, como lector está invitado a reportar cualquier falla para así mejorar este apunte cada vez más. Puede enviar un correo a \href{mailto:cristobalrojas@uc.cl}{cristobalrojas@uc.cl}. ¡Muchas gracias!
% \end{redbox}

% \newpage

\thispagestyle{empty}
\pdfbookmark[section]{\contentsname}{toc}
\tableofcontents

\newpage

\section{Algoritmos para lenguajes libres de contexto}
\subsection{Autómatas apiladores}
\subsubsection{Versión normal}

\fig{img/cap5/idea_automata.png}{0.7}{Idea de un autómata apilador}

\paragraph*{Definición.} Un autómata apilador (\textit{PushDown Automata}, PDA) es una estructura:
\alignformula{
    \ca{P}=(Q,\Sigma,\Gamma,\Delta,q_0,\bot,F)
}
\begin{itemize}
    \item $Q$ es un conjunto finito de \textbf{estados}.
    \item $\Sigma$ es el alfabeto del \textbf{input}.
    \item $q_0 \in Q$ es el estado \textbf{inicial}.
    \item $F$ es el conjunto de estados \textbf{finales}.
    \item $\Gamma$ es el alfabeto de \textbf{stack}.
    \item $\bot \in \Gamma$ es el símbolo \textbf{inicial del stack} (fondo).
    \item $\Delta \subseteq(Q \times(\Sigma \cup\{\epsilon\}) \times \Gamma) \times\left(Q \times \Gamma^*\right)$ es una relación finita de transición.
\end{itemize}

Intuitivamente, la transición:
\alignformula{
    \Big((p,a,A),(q,B_1B_2\cdots B_k)\Big) \in \Delta
}
si el autómata apilador está:
\begin{itemize}
    \item en el estado $p$, leyendo $a$, y en el tope del stack hay una $A$,
\end{itemize}
entonces:
\begin{itemize}
    \item cambia al estado $q$, y modifico el tope $A$ por $B_1B_2\cdots B_k$.
\end{itemize}

Intuitivamente, la transición \textbf{en vacío}:
\alignformula{
    \Big((p,\epsilon,A),(q,B_1B_2\cdots B_k)\Big) \in \Delta
}
si el autómata apilador está:
\begin{itemize}
    \item en el estado $p$, \textit{sin lectura de una letra}, y en el tope del stack hay una $A$,
\end{itemize}
entonces:
\begin{itemize}
    \item cambia al estado $q$, y modifico el tope $A$ por $B_1B_2\cdots B_k$.
\end{itemize}

\ejemplo{}{}{
    $$
        \ca{P}=(Q,\Sigma,\Gamma,\Delta,q_0,\bot,\{q_f\})
    $$
    \begin{itemize}
        \item $Q=\{q_0,q_1,q_f\}$, $\Sigma = \{a,b\}$, $\Gamma = \{A,\bot\}$ y $\Delta$:
              $$
                  \begin{array}{ll}
                      \left(q_0, a, \perp, q_0, A \perp\right)         & q_0 \perp \stackrel{a}{\rightarrow} q_0 A \perp \\
                      \left(q_0, a, A, q_0, A A\right)                 & q_0 A \stackrel{a}{\rightarrow} q_0 A A         \\
                      \left(q_0, b, A, q_1, \epsilon\right)            & q_0 A \stackrel{b}{\rightarrow} q_1             \\
                      \left(q_1, b, A, q_1, \epsilon\right)            & q_1 A \stackrel{b}{\rightarrow} q_1             \\
                      \left(q_1, \epsilon, \perp, q_f, \epsilon\right) & q_1 \perp \stackrel{\epsilon}{\rightarrow} q_f
                  \end{array}
              $$
    \end{itemize}

    \img{img/cap5/ejemplo1.png}{0.65}
}

\paragraph*{Notación.} Dada una palabra $A_1A_2\ldots A_k \in \Gamma^+$ decimos que:
\begin{itemize}
    \item $A_1 A_2 \ldots A_k$ es un stack (contenido),
    \item $A_1$ es el \textbf{tope} del stack y
    \item $A_2 \ldots A_k$ es la \textbf{cola} del stack.
\end{itemize}

\paragraph*{Definición.} Una \textbf{configuración} de $\ca{P}$ es una tupla $(q\cdot \gamma, w) \in (Q\cdot \Gamma^*, \Sigma^*)$ tal que:
\begin{itemize}
    \item $q$ es el estado actual.
    \item $\gamma$ es el contenido del stack.
    \item $w$ es el contenido del input.
\end{itemize}
Decimos que una configuración:
\alignformula{
    (q\cdot \gamma, w) \in (Q\cdot \Gamma^*, \Sigma^*)
}
\begin{itemize}
    \item es \textbf{inicial} si $q\cdot \gamma = q_0\cdot \bot$.
    \item es \textbf{final} si $q\cdot \gamma = q_f\cdot \epsilon$ con $q_f \in F$ y $w=\epsilon$.
\end{itemize}

\paragraph*{Definición.} Se define la relación $\vdash_{\ca{P}}$ de \textbf{siguiente-paso} entre configuraciones de $\ca{P}$:
\alignformula{
    \left(q_1 \cdot \gamma_1, w_1\right) \quad \vdash_{\mathcal{P}} \quad\left(q_2 \cdot \gamma_2, w_2\right)
}
si, y sólo si, existe una transición $\left(q_1, a, A, q_2, \alpha\right) \in \Delta \text { y } \gamma \in \Gamma^*$ tal que:
\begin{itemize}
    \item $w_1 = a \cdot w_2$
    \item $\gamma_1 = A\cdot \gamma$
    \item $\gamma_2 = \alpha \cdot \gamma$
\end{itemize}

Se define $\vdash_{\ca{P}}^*$ como la clausura \textbf{refleja} y \textbf{transitiva} de $\vdash_\ca{P}$. En otras palabras:
\alignformula{
    \begin{gathered}
        \left(q_1 \gamma_1, w_1\right) \vdash_{\mathcal{P}}^*\left(q_2 \gamma_2, w_2\right) \text { si uno puede ir de }\left(q_1 \gamma_1, w_1\right) \text { a }\left(q_2 \gamma_2, w_2\right) \\
        \text { en } 0 \text { o más pasos. }
    \end{gathered}
}
\ejemplo{}{}{
    Para la palabra $w=aaabbb$, tenemos la ejecución:
    \img{img/cap5/ejemplo2.png}{0.7}
}

\paragraph*{Definiciones.} $\cal{P}$ \textbf{acepta} $w$ si, y sólo si, $\left(q_0 \perp, w\right) \vdash_{\mathcal{P}}^*\left(q_f, \epsilon\right)$ para algún $q_f \in F$.

\hspace{70pt} El \textbf{lenguaje aceptado} por $\ca{P}$ se define como:
\alignformula{
    \ca{L}(\ca{P})=\{w\in \Sigma^*\| \ \ca{P} \text{ acepta } w\}
}
\ejemplo{}{}{
    El lenguaje aceptado por el PDA utilizado en los ejemplos anteriores es $\ca{L}(\ca{P})=\{a^nb^n\ | \ n \ge 0\}$.
}

\subsubsection{Versión alternativa}
Esta definición de autómata apilador es poco común pero trae algunas ventajas:
\begin{itemize}
    \item Es un modelo que ayuda a entender mejor los algoritmos de evaluación para gramáticas.
    \item Es un modelo menos estándar pero mucho más sencillo.
    \item Al profe Cristian le gustó y lo encontró interesante.
\end{itemize}

\paragraph*{Definición.} Un \textbf{PDA alternativo} es una estructura:
\alignformula{
    \ca{D}=(Q,\Sigma,\Delta,q_0,F)
}
\begin{itemize}
    \item $Q$ es un conjunto finito de \textbf{estados}.
    \item $\Sigma$ es el alfabeto del \textbf{input}.
    \item $q_0 \in Q$ es el estado \textbf{inicial}.
    \item $F$ es el conjunto de estados \textbf{finales}.
    \item $\Delta \subseteq Q^+ \times (\Sigma \cup \{\epsilon\})\times Q^*$ es una \textbf{relación finita de transición}.
\end{itemize}
Intuitivamente, la transición:
\alignformula{
    \Big( A_1\ldots A_i, a, B_1 \ldots B_j \Big) \in \Delta
}
si el autómata apilador tiene:
\begin{itemize}
    \item $A_1\ldots A_i$ en el tope del stack y leyendo $a$,
\end{itemize}
entonces:
\begin{itemize}
    \item cambia el tope $A_1\ldots A_i$ por $B_1\ldots B_j$.
\end{itemize}

En este tipo de autómata apilador, \textbf{no hay diferencia} entre estados y alfabeto del stack.

\paragraph*{Definición.} Una \textbf{configuración} de $\ca{D}$ es una tupla
\alignformula{
    (q_1\ldots q_k, w) \in (Q^+,\Sigma^*)
}
tal que:
\begin{itemize}
    \item $q_1\ldots q_k$ es el contenido del stack con $q_1$ el tope del stack.
    \item $w$ es el contenidod el input.
\end{itemize}
Decimos que una configuración:
\begin{itemize}
    \item $(q_0,w)$ es \textbf{inicial}.
    \item $(Q_f,\epsilon)$ es \textbf{final} si $q_f \in F$.
\end{itemize}

\paragraph*{Definición.} Se define la relación $\vdash_{\ca{D}}$ de \textbf{siguiente-paso} entre configuraciones de $\ca{D}$:
\alignformula{
    \left( \gamma_1, w_1\right) \quad \vdash_{\mathcal{D}} \quad\left(\gamma_2, w_2\right)
}
si, y sólo si, existe una transición $\left(\alpha, a, \beta\right) \in \Delta \text { y } \gamma \in \Gamma^*$ tal que:
\begin{itemize}
    \item $w_1 = a \cdot w_2$
    \item $\gamma_1 = \alpha\cdot \gamma$
    \item $\gamma_2 = \beta \cdot \gamma$
\end{itemize}

Se define $\vdash_{\ca{D}}^*$ como la clausura \textbf{refleja} y \textbf{transitiva} de $\vdash_\ca{D}$.

\paragraph*{Definiciones.} $\ca{D}$ \textbf{acepta} $w$ si, y sólo si, $(q_0,w) \vdash_\ca{D}^* (q_f,\epsilon)$ para algún $q_f \in F$. Además, el \textbf{lenguaje aceptado} por $\ca{D}$ se define como:
\alignformula{
    \ca{L}(\ca{D})=\{w\in \Sigma^*\| \ \ca{D} \text{ acepta } w\}
}

\newpage
\ejemplo{}{}{
    $$
        \ca{D}=(Q,\{a,b\},\Delta,q_0,F)
    $$
    \begin{itemize}
        \item $Q=\{\bot, q_0, q_1, q_f\}$ y $\Delta$:
              \img{img/cap5/ejemplo4.png}{0.6}
    \end{itemize}
    $$
        \ca{L}(\ca{D})=\{a^nb^n \ |\ n\ge 1\}
    $$
}

\teorema{}{}{
    Para todo autómata apilador $\ca{P}$ existe un autómata apilador alternativo $\ca{D}$, y viceversa, tal que:
    $$
        \ca{L}(\ca{P}) = \ca{L}(\ca{D})
    $$
}
El teorema anterior nos dice que podemos usar ambos modelos de manera \textbf{equivalente}.

\subsection{Autómatas apiladores vs gramáticas libres de contexto}
¿En qué se parecen CFG a PDA?
\fig{img/cap5/cfg_vs_pda.png}{0.3}{Gramáticas vs Autómatas apiladores}

\teorema{}{}{
    Todo \textbf{lenguaje libre de contexto} puede ser descrito equivalentemente por:
    \begin{itemize}
        \item Una gramática libre de contexto (\textbf{CFG}).
        \item Un autómata apilador (\textbf{PDA}).
    \end{itemize}
}

\subsubsection{Desde CFG a PDA}
Partimos enunciado un teorema:
\teorema{}{}{
    Para toda gramática libre de contexto $\ca{G}$, existe un \textbf{autómata apilador alternativo} $\ca{D}$, tal que:
    $$
        \ca{L}(\ca{G}) = \ca{L}(\ca{D})
    $$
}

\paragraph*{Construcción $\ca{D}$ desde $\ca{G}$.} Sea $\ca{G}=(V,\Sigma,P,S)$ una CFG. Construimos un PDA alternativo $\ca{D}$ que acepta $\ca{L}(\ca{G})$:
\alignformula{
    \ca{D}=\Big( V \cup \Sigma \cup \{q_0,q_f\}, \Sigma, \Delta, q_0, \{q_f\} \Big)
}
La relación de transición $\Delta$ se define como:
\begin{table}[H]
    \centering
    \begin{tabular}{lllll}
        $\Delta$ & $=$ & $\{ (q_0, \epsilon, S \cdot q_f) \}$               & $\cup$ &                     \\
                 &     & $\{ (X,\epsilon,\gamma)\ | \ X\to \gamma \in P \}$ & $\cup$ & \textbf{(Expandir)} \\
                 &     & $\{ (a,a,\epsilon) \ | \ a \in \Sigma \}$          &        & \textbf{(Reducir)}  \\
                 &     &                                                    &        &
    \end{tabular}
\end{table}
\paragraph*{Demostración $\ca{L}(\ca{G}) = \ca{L}(\ca{D})$.} Debemos demostrar dos direcciones: $\ca{L}(\ca{G}) \subseteq \ca{L}(\ca{D})$ y $\ca{L}(\ca{D}) \subseteq \ca{L}(\ca{G})$.

\paragraph*{Demostración $\ca{L}(\ca{G}) \subseteq \ca{L}(\ca{D})$.} Para cada $w \in \ca{L}(\ca{G})$ debemos encontrar una ejecución de aceptación de $\ca{D}$ sobre $w$. ¿Cómo encontramos esta ejecución? La idea es que para cada árbol de derivación $\ca{T}$ de $\ca{G}$ sobre $w$, construimos una ejecución de $\ca{D}$ sobre $w$ que recorre el árbol $\ca{T}$ \textbf{en profundidad} (DFS). Por tanto, debemos usar \textbf{inducción} sobre la altura del árbol $\ca{T}$.

\paragraph*{Hipótesis de inducción.} Para todo árbol de derivación $\ca{T}$ de $\ca{G}$ con \textbf{altura} $h$ tal que:
\begin{itemize}
    \item la raíz de $\ca{T}$ es $X$, y
    \item $\ca{T}$ produce la palabra $w$
\end{itemize}
entonces $(X\cdot\gamma, w) \vdash_\ca{D}^* (\gamma, \epsilon)$ para todo $\gamma \in Q^+$.

\paragraph{Caso base: $h=1$.} Si $\ca{T}$ tiene altura $1$, entonces:
\begin{itemize}
    \item $\ca{T}$ produce la palabra $w=a$ para algún $a\in \Sigma$ y
    \item $\ca{T}$ consiste de un nodo $X$ y un hijo $a$ con $X \to a$.
\end{itemize}
Entonces para todo $\gamma \in Q^+$:
$$
    (X \cdot \gamma, a) \vdash_\mathcal{D} (a \cdot \gamma, a) \vdash_\mathcal{D}(\gamma, \epsilon)
$$
es una ejecución de $\ca{D}$ sobre $a$.

\paragraph*{Caso inductivo: $h=n$.} Suponemos que el árbol de derivación $\ca{T}$ de $\ca{G}$ tiene \textbf{altura} $n$ tal que:
\begin{itemize}
    \item la raíz de $\ca{T}$ es $X$, y
    \item $\ca{T}$ produce la palabra $w$.
\end{itemize}
\textbf{Sin pérdida de generalidad}, suponga que $\ca{T}$ es de la forma:
\img{img/cap5/dem1.png}{0.5}
donde $w = u\cdot v$ y $X\to YZ$. Por HI, se tiene que para todo $\gamma_1, \gamma_2 \in Q^+$:
$$
    \begin{aligned}
        \left(Y \cdot \gamma_1, u\right) & \vdash_{\mathcal{D}}^*\left(\gamma_1, \epsilon\right) \\
        \left(Z \cdot \gamma_2, v\right) & \vdash_{\mathcal{D}}^*\left(\gamma_2, \epsilon\right)
    \end{aligned}
$$
Para $\gamma \in Q^+$ \textbf{construimos} la siguiente ejecución de $\ca{D}$ sobre $w=uv$:
$$
    (X \cdot \gamma, u v) \vdash_{\mathcal{D}}(Y Z \cdot \gamma, u v) \vdash_{\mathcal{D}}^*(Z \cdot \gamma, v) \vdash_{\mathcal{D}}^*(\gamma, \epsilon)
$$
\hfill $\blacksquare$

La demostración de $\ca{L}(\ca{D}) \subseteq \ca{L}(\ca{G})$ se deja como ejercicio propuesto al lector.

% \paragraph*{Demostración $\ca{L}(\ca{D}) \subseteq \ca{L}(\ca{G})$.} Para cada $w \in \ca{L}(\ca{D})$ debemos encontrar un árbol de derivación de $\ca{G}$ para $w$. ¿Cómo encontramos un árbol de derivación para $w$? La idea es que si tenemos una ejecución de $\ca{D}$ sobre $w$ de la forma:
% $$
%     \left(X \cdot q_f, w\right) \vdash_{\mathcal{D}}^*\left(q_f, \epsilon\right)
% $$
% entonces $X \underset{\mathcal{G}}{\stackrel{\star}{\Rightarrow}} w$. Por tanto, podemos usar \textbf{inducción} en la cantidad de pasos de la ejecución.

% \paragraph{Hipótesis de inducción.} Para toda ejecución de $\ca{D}$ sobre $w$ de largo $k$ de la forma:
% $$
%     \left(X \cdot q_f, w\right)=\left(\gamma_0, w_0\right) \vdash_\mathcal{D}\left(\gamma_1, w_1\right) \vdash_\mathcal{D} \cdots \vdash_\mathcal{D}\left(\gamma_k, w_k\right)=\left(q_f, \epsilon\right)
% $$
% entonces $X \underset{\mathcal{G}}{\stackrel{\star}{\Rightarrow}} w$.

\subsubsection{Desde PDA a CFG}
Partimos enunciando el siguiente teorema:
\teorema{}{}{
    Para todo autómata apilador $\ca{P}$, existe una gramática libre de contexto $\ca{G}$ tal que:
    $$
        \ca{L}(\ca{P}) = \ca{L}(\ca{G})
    $$
}
\paragraph{Demostración $\ca{L}(\ca{P}) = \ca{L}(\ca{G})$.} Sea $\ca{P}=(Q,\Sigma,\Gamma,\Delta,q_0,\bot,F)$ un PDA (normal). Los pasos a seguir son:
\begin{enumerate}
    \item Convertir $\ca{P}$ a un PDA $\ca{P}'$ con \textbf{UN solo estado}.
    \item Convertir $\ca{P}'$ a una gramática libre de contexto $\ca{G}$.
\end{enumerate}

\paragraph{Paso 1.} Sea $\ca{P}=(Q,\Sigma,\Gamma,\Delta,q_0,\bot,F)$ un PDA. Podemos analizar:
\begin{itemize}
    \item ¿Por qué NO necesitamos la información de los estados?
    \item ¿Cómo guardamos la información de los estados en el stack?
\end{itemize}

Esto conlleva a la siguiente pregunta: \textit{Si el PDA está en el estado $p$ y en el tope del stack hay una $A$, ¿a cuál estado llegaré al remover $A$ del stack?} \medbreak

La solución a esta pregunta es que podemos \textbf{adivinar} (no-determinismo) el estado que vamos a llegar cuando removamos $A$ del stack. \medbreak

\textbf{Sin pérdida de generalidad}, podemos asumir que
\begin{enumerate}
    \item Todas las transiciones son de la forma:
          $$
              q A \stackrel{c}{\rightarrow} p B_1 B_2 \quad \text{o} \quad q A \stackrel{c}{\rightarrow} p \epsilon
          $$
          con $c \in (\Sigma \cup \{e\})$.
    \item Existe $q_f \in Q$ tal que si $w \in \ca{L}(\ca{P})$ entonces:
          $$
              \left(q_0 \bot, w\right) \vdash_{\mathcal{D}}^*\left(q_f, \epsilon\right)
          $$
\end{enumerate}
Estos dos puntos nos aseguran  que siempre llegamos al \textbf{mismo estado} $q_f$. Luego, construimos el autómata apilador $\ca{P}'$ con \textbf{un solo estado}:
$$
    \mathcal{P}^{\prime}=\left(\{q\}, \Sigma, \Gamma^{\prime}, \Delta^{\prime},\{q\}, \perp^{\prime},\{q\}\right)
$$
\begin{itemize}
    \item $\Gamma'= Q\times \gamma \times Q$.

          \textit{``$(p, A, q) \in \Gamma'$ si desde $p$ leyendo $A$ en el tope del stack llegamos a $q$ al hacer pop de $A$''.}

    \item $\bot' = (q_0,\bot, q_f)$.

          \textit{``El autómata parte en $q_0$ y al hacer pop de $\bot$ llegará a $q_f$''.}

    \item Si $p A \stackrel{c}{\rightarrow} p^{\prime} B_1 B_2 \in \Delta$ con $c \in (\Sigma \cup \{\epsilon\})$, entonces \textbf{para todo} $p_1,p_2 \in Q$:
          $$
              q\left(p, A, p_2\right) \stackrel{c}{\rightarrow} q\left(p^{\prime}, B_1, p_1\right)\left(p_1, B_2, p_2\right) \in \Delta^{\prime}
          $$

    \item Si $p A \stackrel{c}{\rightarrow} p^{\prime}\in \Delta$ con $c \in (\Sigma \cup \{\epsilon\})$, entonces:
          $$
              q\left(p, A, p^{\prime}\right) \stackrel{c}{\rightarrow} q \in \Delta^{\prime}
          $$
\end{itemize}

\paragraph{Hipótesis de inducción (en el número de pasos $n$).} Para todo $p,p' \in Q$, $A \in \Gamma$ y $w \in \Sigma^*$ se cumple que:
$$
    (p A, w) \vdash_{\mathcal{P}}^n\left(p^{\prime}, \epsilon\right) \quad \text { si, y solo si, } \quad\left(q\left(p, A, p^{\prime}\right), w\right) \vdash_{\mathcal{P}^{\prime}}^n(q, \epsilon)
$$
donde $\vdash_\ca{P}^n$ es la relación de \textbf{siguiente-paso} de $\ca{P}$ $n$-veces. \medbreak

Si demostramos esta hipótesis, habremos demostrado que $\ca{L}(\ca{P}) = \ca{L}(\ca{P'})$. ¿Por qué?

\paragraph{Caso base: $n=1$.} Para todo $p,p' \in Q$, y $A \in \Gamma$ se cumple que:
$$
    (p A, c) \vdash_\mathcal{P}\left(p^{\prime}, \epsilon\right) \quad \text { si, y solo si, } \quad\left(q\left(p, A, p^{\prime}\right), c\right) \vdash_{\mathcal{P}^{\prime}}(q, \epsilon)
$$
para todo $c \in (\Sigma \cup \{\epsilon\})$.

\paragraph{Caso inductivo.} \textbf{Sin pérdida de generalidad}, suponga que $pA \overset{a}{\to} p_1A_1A_2$ y $w=auv$, entonces

$$
    (p A, \underbrace{a u v}_w) \vdash_{\mathcal{P}}^n\left(p^{\prime}, \epsilon\right) \text { ssi }(p A, a u v) \vdash_{\mathcal{P}}\left(p_1 A_1 A_2, u v\right) \vdash_{\mathcal{P}}^i\left(p_2 A_2, v\right) \vdash_{\mathcal{P}}^j\left(p^{\prime}, \epsilon\right)
$$
\begin{align*}
     & \text{ssi }  \left(p_1 A_1, u\right) \vdash_{\mathcal{P}}^i\left(p_2, \epsilon\right) \text { y } \quad\left(p_2 A_2, v\right) \vdash_{\mathcal{P}}^j\left(p^{\prime}, \epsilon\right)                                  \\
     & \text {ssi }  \left(q\left(p_1, A_1, p_2\right), u\right) \vdash_{\mathcal{P}^{\prime}}^i(q, \epsilon) \text { y } \quad\left(q\left(p_2, A_2, p^{\prime}\right), v\right) \vdash_{\mathcal{P}^{\prime}}^j(q, \epsilon) \\
     & \text {ssi }  \left.\left(q\left(p, A, p^{\prime}\right), auv\right) \vdash_{\mathcal{P}}\left(q\left(p_1, A_1, p_2\right)\left(p_2, A_2, q\right)\right), u v\right) \vdash_{\mathcal{P}}^{i+j}(q, \epsilon)
\end{align*}

\hfill $\blacksquare$

\paragraph{Paso 2.} Sea $\ca{P}=(\{q\},\Sigma,\Gamma,\Delta,q,\bot,\{q\})$ un PDA con \textbf{UN solo estado}. Contruimos la gramática:
$$
    \ca{G} = (V, \Sigma, P, \bot)
$$
\begin{itemize}
    \item $V=\gamma$.
    \item Si $qA \overset{\epsilon}{\to} q\alpha \in \Delta$ entonces $A \to \alpha \in P$
    \item Si $qA \overset{a}{\to} q\alpha \in \Delta$ entonces $A \to a\alpha \in P$
\end{itemize}
La demostración de este paso queda como ejercicio propuesto al lector.

% \section{Propiedades de lenguajes regulares}

\subsection{Lema de bombeo}

Supongamos que deseamos aceptar el siguiente lenguaje:
$$
    L =\left\{a^i b^i \mid i \geq 0\right\} = \{\epsilon, ab, aabb, aaabbb, aaaaabbbb, \ldots\}
$$

con un \textbf{autómata finito determinista}. ¿Es posible? La respuesta es que no, ya que nuestros autómatas no tienen la capacidad de ``contar''. Por ende, $L$ sería un lenguaje NO regular ya que no podría ser definido por un autómata.

\paragraph{Enunciado.} Sea $L \subseteq \Sigma^*$. Si $L$ es \textbf{regular}, entonces:
\alignformula{
    \text{(LB)} \quad   &\text{existe un }  N>0 \text{ tal que} \\
    &\text{para toda palabra } x \cdot y \cdot z \in L \text{ con } |y| \geq N \\
    &\text{existen palabras } u \cdot v \cdot w=y \text{ con } v \neq \epsilon \text{ tal que} \\
    &\text{para todo } i \geq 0, \quad x \cdot u \cdot v^i \cdot w \cdot z \in L.
}

El contrapositivo del lema de bombeo nos servirá para demostrar que un lenguaje $L$ NO es regular. \medbreak

Sea $L \subseteq \Sigma^*$. Si:

\alignformula{
    \text{(}\neg\text{LB)} \quad   &\text{para todo }  N>0 \\
    &\text{existe una palabra } x \cdot y \cdot z \in L \text{ con } |y| \geq N \text{ tal que}\\
    &\text{para todo } u \cdot v \cdot w=y \text{ con } v \neq \epsilon \\
    &\text{existe un } i \geq 0, \quad x \cdot u \cdot v^i \cdot w \cdot z \notin L. \\
    &\text{entonces } L \text{ NO es regular.}
}
\img{img/cap2/demonio.png}{0.3}

\paragraph{LB versión juego.} \textit{``Dado un lenguaje $L \subseteq \Sigma^*$, si \textbf{UNO} tiene una estrategia ganadora en el juego ($\neg$LB) para toda estrategia posible del demonio, entonces L \textbf{NO es} regular''.} Con \textbf{estrategia}, nos referimos a todas las movidas posibles que podría ejecutar el \textbf{demonio} (considerar todos los casos posibles de sus elecciones).


\ejemplo{}{}{
    Considere el lenguaje $L = \{a^i b^i \mid i \geq 0\}$:
    \img{img/cap2/ejemplo1.png}{0.175}

    Ganamos el juego ya que con $i = 2$ estaremos bombeando más $b$-letras y entonces la palabra no tendrá la misma cantidad de $a$-letras que de $b$-letras ($i \neq j$), por ende, $L$ NO es regular.
}


\ejemplo{}{}{
    Considere el lenguaje $L = \{a^n b^m \mid n \geq m\}$:
    \img{img/cap2/ejemplo2.png}{0.175}

    Ganamos el juego ya que, nuevamente, con $i = 2$, estaremos bombeando más $b$-letras y entonces la palabra puede tener más $b$-letras que $a$-letras ($n < m$), y así $L$ NO es regular.
}

\newpage

\ejemplo{}{}{
    Considere el lenguaje $L = \{w \cdot w \mid w \in \{a,b\}^*\}$
    \img{img/cap2/ejemplo3.png}{0.175}

    Ganamos el juego ya que con la elección de $i = 0$ estamos haciendo que una de las mitades de la palabra sea distinta a su otra mitad, por ende, $L$ NO es regular.
}

\ejemplo{}{}{
    Considere el lenguaje $L = \{a^{2^n} \mid n > 0\}$
    \img{img/cap2/ejemplo4.png}{0.175}

    Ganamos el juego ya que con la elección de $i = 2$, tenemos que en la elección de $y$ tendremos una mayor cantidad de $a$-letras bombeadas y se romperá el equilibrio $2^N - N + N$, por ende, $L$ NO es regular.
}

\newpage

\subsection{Minimización de autómatas}

¿Cómo minimizamos un autómata finito?
\begin{figure}[H]
    \centering
    \includegraphics[scale=0.4]{img/cap2/min.png}
    \includegraphics[scale=0.1755]{img/cap2/min_2.png}
    \includegraphics[scale=0.35]{img/cap2/min_3.png}
    \caption{Idea de minimización}
\end{figure}

\begin{enumerate}
    \item Eliminar estados inaccesibles.
          \begin{itemize}
              \item Fácil de realizar y no cambia el lenguaje del autómata finito.
          \end{itemize}
    \item Colapsar estados ``equivalentes''.
          \begin{itemize}
              \item ¿Cómo sabemos cuáles estados colapsar y cúales no?
          \end{itemize}
\end{enumerate}

\ejemplo{}{}{
    Considere el siguiente autómata:
    \img{img/cap2/ejemplo5_1.png}{0.4}
    Podemos:
    \begin{itemize}
        \item Colapsar estados 1 y 2.
        \item Colapsar estados 3 y 4.
    \end{itemize}
    \begin{figure}[H]
        \centering
        \includegraphics[scale=0.1755]{img/cap2/ejemplo5_2.png}
        \includegraphics[scale=0.4]{img/cap2/ejemplo5_3.png}
    \end{figure}
}

\ejemplo{}{}{
    Considere el siguiente autómata:
    \img{img/cap2/ejemplo6_1.png}{0.4}
    Podemos:
    \begin{itemize}
        \item Colapsar estados 1 y 2.
        \item Colapsar estados 3, 4 y 5.
    \end{itemize}
    \begin{figure}[H]
        \centering
        \includegraphics[scale=0.1755]{img/cap2/ejemplo6_2.png}
        \includegraphics[scale=0.4]{img/cap2/ejemplo6_3.png}
    \end{figure}
}

\subsubsection{Colapsar estados}

\paragraph{Definición.} Sea $\ca{A} = (Q, \Sigma, \delta, q_0, F)$ un DFA.

Se define la \textbf{función de transición extendida} $\hat{\delta}: Q \times \Sigma^* \rightarrow Q$ inductivamente como:
\alignformula{
    \begin{array}{rll}
        \hat{\delta}(q, \epsilon)  & \stackrel{\text { def }}{\equiv} & q                             \\
        \hat{\delta}(q, w \cdot a) & \stackrel{\text { def }}{\equiv} & \delta(\hat{\delta}(q, w), a)
    \end{array}
}

\paragraph{Definición.} Decimos que $p$ y $q$ son \textbf{indistingibles} ($p \approx_\ca{A} q$) si:
\alignformula{
    p \approx_\mathcal{A} q \quad \text { ssi } \quad(\hat{\delta}(p, w) \in F \Leftrightarrow \hat{\delta}(q, w) \in F), \text { para todo } w \in \Sigma^* .
}
Decimos que $p$ y $q$ son \textbf{distingibles} si NO son indistingibles ($p \not\approx_\ca{A} q$).

\paragraph{Recordatorio relaciones de equivalencia.} Una relación $\approx_R$ sobre un conjunto $X$ se dice de \textbf{equivalencia} si es:
\begin{itemize}
    \item \textbf{Refleja:} $\forall p \in X.\ p \approx_R p$
    \item \textbf{Simétrica:}  $\forall p,q \in X.$ si $p \approx_R q$ entonces $q \approx_R p$.
    \item \textbf{Transitiva:} $\forall p,q,r \in X.$ si $p \approx_R q$ y $q \approx_R r$, entonces $p \approx_R r$.
\end{itemize}

Para un elemento $p \in X$ se define su \textbf{clase de equivalencia} según $\approx_R$ como:
$$
    [p]_{\approx_R} = \{q \mid q \approx_R p\}
$$
Una función $f: X \to X$ se dice \textbf{bien definida} sobre $\approx_R$ si:
$$
    p \approx_R q \quad \text{entonces} \quad f(p) \approx_R f(q)
$$

\paragraph{Propiedades de $\approx_\ca{A}$.} Tenemos que:
\begin{itemize}
    \item $\approx_\ca{A}$ es una \textbf{relación de equivalencia}, es decir, es refleja, simétrica y transitiva.
    \item Cada estado $p \in Q$ esta en exactamente una clase de equivalencia:
          $$
              [p]_{\approx_\ca{A}} = \{q \mid q \approx_\ca{A} p\}
          $$
    \item Para todo $a \in \Sigma$ la función $\delta(\cdot, a): Q \to Q$ esta \textbf{bien definida} sobre $\approx_\ca{A}$:
          $$
              p \approx_\ca{A} q \quad \text{entonces} \quad \delta(p,a) \approx_\ca{A} \delta(q,a)
          $$
\end{itemize}

\paragraph{El autómata cuociente.} Para un DFA $\ca{A} = (Q,\Sigma,\delta,q_0,F)$ se define el DFA:
\alignformula{
    \ca{A} / \approx \;= (Q_\approx, \Sigma, \delta_\approx, q_\approx, F_\approx)
}
\begin{itemize}
    \item $Q_{\approx}=\left\{[p]_{\approx_\mathcal{A}} \mid p \in Q\right\}$
    \item $\delta_{\approx}\left([p]_{\approx_\mathcal{A}}, a\right)=[\delta(p, a)]_{\approx_\mathcal{A}}$
    \item $q_{\approx}=\left[q_0\right]_{\approx_\mathcal{A}}$
    \item $F_{\approx}=\left\{[p]_{\approx_\mathcal{A}} \mid p \in F\right\}$
\end{itemize}

\teorema{}{}{
    Para todo autómata finito determinista $\ca{A}$ se cumple que:
    $$
        \ca{L}(\ca{A}) = \ca{L}(\ca{A} / \approx)
    $$
}

\paragraph{Demostración.} Queda como ejercicio propuesto al lector.

\subsubsection{Algoritmo de minimización}

El objetivo es buscar los pares de estados que son \textbf{distingibles}:
\begin{enumerate}
    \item Construya una tabla con los pares $\{p,q\}$ inicialmente sin marcar.
    \item Marque $\{p,q\}$ si $p \in F$ y $q \notin F$ o viceversa.
    \item Repita este paso hasta que no hayan más cambios:
          \begin{itemize}
              \item Si $\{p,q\}$ no están marcados y $\{\delta(p,a),\delta(q,a)\}$ estan marcados para algún $a \in \Sigma$, entonces marque $\{p,q\}$.
          \end{itemize}
    \item Al terminar, $p \not\approx_\ca{A} q$ si, y sólo si, la entrada $\{p,q\}$ está marcada.
\end{enumerate}

Veamos como funciona el algoritmo. Considere el siguiente autómata y una tabla que relacione todos los pares de estados:

\img{img/cap2/amin1.png}{0.175}

\begin{multicols}{2}
    \begin{enumerate}

        \item Construya una tabla con los pares $\{p,q\}$ inicialmente sin marcar.
              \img{img/cap2/amin2.png}{0.175}

        \item Marque $\{p,q\}$ si $p \in F$ y $q \not\in F$ o viceversa.
              \img{img/cap2/amin3.png}{0.175}
    \end{enumerate}
\end{multicols}

\begin{enumerate}
    \item[3.] Repita este paso hasta que no hayan más cambios:
        \begin{itemize}
            \item Si $\{p,q\}$ no están marcados y $\{\delta(p,a), \delta(q,a)\}$ estan marcados para algún $a \in \Sigma$, entonces marque $\{p,q\}$.
        \end{itemize}
        \begin{figure}[H]
            \centering
            \includegraphics[scale=0.175]{img/cap2/amin4.png}
            \includegraphics[scale=0.175]{img/cap2/amin5.png}
            \includegraphics[scale=0.175]{img/cap2/amin6.png}
            \includegraphics[scale=0.175]{img/cap2/amin7.png}
            \includegraphics[scale=0.175]{img/cap2/amin8.png}
            \includegraphics[scale=0.175]{img/cap2/amin9.png}
        \end{figure}
\end{enumerate}

\begin{enumerate}
    \item[4.] Al terminar, $p \not\approx_\ca{A} q$ si, y sólo si, la entrada $\{p,q\}$ está marcada.
\end{enumerate}

Así, vemos que los pares indistingibles son todas las entradas \textbf{NO} marcadas.

\subsection{Teorema de Myhill-Nerode}

La sección anterior deja muchas incógnitas:
\begin{enumerate}
    \item ¿Cómo sabemos si el autómata del algoritmo es un \textbf{mínimo}?
    \item Dado $L$, ¿existe un \textbf{único} autómata mínimo?
    \item Dado un $\ca{A}$, ¿es posible \textbf{construir} un autómata mínimo equivalente?
\end{enumerate}

En esta sección, demostraremos que:
\begin{itemize}
    \item El autómata con el mínimo de estados es \textbf{único}.
    \item El algoritmo de minimización \textbf{siempre} construye el autómata mínimo.
\end{itemize}

\paragraph{Estrategia.} Para demostrar lo dicho anteriormente, seguimos los siguientes pasos:
\begin{enumerate}
    \item Desde un DFA $\ca{A}$, definiremos una relación de equivalencia (RE) $\equiv_\ca{A}$ entre palabras en $\Sigma^*$.
    \item Desde una RE $\equiv$ entre palabras, construiremos un DFA $\ca{A}_\equiv$.
    \item A partir de un lenguaje $L$, definiremos una RE $\equiv_L$.
    \item $\ca{A}_{\equiv_L}$ define el autómata con la \textbf{menor cantidad de estados}.
    \item $\ca{A}_{\equiv_L}$ es equivalente al resultado de nuestro \textbf{algoritmo de minimización}.
\end{enumerate}

\subsubsection{Relaciones de Myhill-Nerode}

Sea $L \subseteq \Sigma^*$ cualquier lenguaje.

\paragraph{Definición.} Una relación de equivalencia $\equiv$ en $\Sigma^*$ es de \textbf{Myhill-Nerode} para $L$ si:
\begin{enumerate}
    \item $\equiv$ es una \textbf{congruencia por la derecha}.
    \item $\equiv$ \textbf{refina} $L$.
    \item El número de clases de equivalencia de $\equiv$ es \textbf{finita}.
\end{enumerate}

A partir de una relación $\equiv$ de Myhill-Nerode podemos construir un DFA $\ca{A}_\equiv$.
\alignformula{
    \ca{A} \quad &\Rightarrow \quad \equiv_\ca{A} \\
    \equiv \quad &\Rightarrow \quad \ca{A}_\equiv
}

\paragraph{Construcción del DFA $\ca{A}_\equiv$.} Dada una relación de Myhill-Nerode $\equiv$ para $L \subseteq \Sigma^*$, definimos el autómata:
\alignformula{
    \ca{A}_\equiv = (Q_\equiv, \Sigma, \delta_\equiv, q_\equiv, F_\equiv)
}
\begin{itemize}
    \item $Q_{\equiv}=\left\{[w]_{\equiv} \mid w \in \Sigma^*\right\}$
    \item $q_{\equiv}=[\epsilon]_{\equiv}$
    \item $F_{\equiv}=\left\{[w]_{\equiv} \mid w \in L\right\}$
    \item $\delta_{\equiv}\left([w]_{\equiv}, a\right)=[w a]_{\equiv}$
\end{itemize}

\teorema{}{}{
    Cada cualquier $L \subseteq \Sigma^*$, tenemos que
    $$
        \ca{L}(\ca{A}_\equiv) = L
    $$
}

Podemos establecer que $\ca{A} \; \Rightarrow \; \equiv_\ca{A}$ y $\equiv \; \Rightarrow \; \ca{A}_\equiv$ son procesos inversos, conclusión que se ilustra en el siguiente teorema.

\teorema{}{}{
    \begin{enumerate}
        \item Si $\ca{A}$ es un DFA que acepta $L$ y si construimos:
              $$
                  \ca{A} \quad \Rightarrow \quad \equiv_\ca{A} \quad \Rightarrow \quad \ca{A}_{\equiv_\ca{A}}
              $$
              entonces $\ca{A}$ es \textbf{isomorfo} (``equivalente'') a $\ca{A}_{\equiv_\ca{A}}$.

        \item Si $\equiv$ es una relación de Myhill-Nerode para $L$ y si construimos:
              $$
                  \equiv \quad \Rightarrow \quad \ca{A}_\equiv \quad \Rightarrow \quad \equiv_{\ca{A}_\equiv}
              $$
              entonces la relación $\equiv$ es \textbf{equivalente} a $\equiv_{\ca{A}_\equiv}$.
    \end{enumerate}
}

La demostración de ambos teoremas queda propuesto como ejercicio al lector.

\subsubsection{Camino al teorema}
Antes de enunciar el Teorema de Myhill-Nerode, debemos aún mencionar algunas definiciones.

\paragraph{Definición.} Dado un lenguaje $L \subseteq \Sigma^*$, se define la relación de equivalencia $\equiv_L$ como:
\alignformula{
    u \equiv_L v \quad \text{ssi} \quad(u \cdot w \in L \Leftrightarrow v \cdot w \in L) \quad \forall w \in \Sigma^*
}

\ejemplo{}{}{
    Sea $L = (ab)^*$. Algunas clases de equivalencia para $L$ son:
    \begin{itemize}
        \item $[\epsilon]_{\equiv_L}=\{\epsilon, a b, a b a b, a b a b a b, \ldots\}$.
        \item $[a]_{\equiv_L}=\{a, a b a, a b a b a, a b a b a b a, \ldots\}$.
        \item $[b]_{\equiv_L}=\{b, b b, b a, a b b, \ldots\}$
    \end{itemize}
}

\paragraph{Propiedades.} $\equiv_L$ se caracteriza por:
\begin{enumerate}
    \item Ser una \textbf{congruencia por la derecha}:
          $$
              u \equiv_L v \text { entonces } u \cdot w \equiv_L v \cdot w \quad \forall w \in \Sigma^*
          $$
    \item Refinar a $L$:
          $$
              u \equiv_L v \text { entonces }(u \in L \Leftrightarrow v \in L)
          $$
    \item Si $\equiv$ es una congruencia por la derecha y refina $L$, entonces $\equiv$ \textbf{refina} a $\equiv_L$:
          $$
              u \equiv v \text { entonces } u \equiv_L v
          $$
\end{enumerate}

Con todo lo anterior, estamos en condiciones de enunciar el teorema.

\teorema{}{}{
    Sea $L \subseteq \Sigma^*$. Las siguientes propiedades son equivalentes:
    \begin{enumerate}
        \item $L$ es \textbf{regular}.
        \item Existe una \textbf{relación de Myhill-Nerode} para $L$.
        \item La relación $\equiv_L$ tiene una cantidad \textbf{finita} de clases de equivalencia.
    \end{enumerate}
}

\paragraph{Demostración teorema 10.} Del punto 1 al 2, tenemos que si $L$ es regular, entonces:
\begin{itemize}
    \item existe un autómata finito $\ca{A}$ tal que $L = \ca{L}(\ca{A})$.
    \item $\equiv_\ca{A}$ es una relación de Myhill-Nerode para $L$.
\end{itemize}

Del punto 2 al 3, sea $\equiv$ una relación de Myhill-Nerode para $L$, entonces:
\begin{itemize}
    \item $\equiv$ tiene una cantidad finita de clases de equivalencia.
    \item $\equiv_L$ tiene una cantidad finita de clases de equivalencia.
\end{itemize}

Del punto 3 al 1, si $\equiv_L$ tiene una cantidad \textbf{finita} de clases de equivalencia, entonces:
\begin{itemize}
    \item $\equiv_L$ es una relación de Myhill-Nerode para L.
    \item $\ca{A}_{\equiv_L}$ es un autómata finito para $L$. \hfill $\blacksquare$
\end{itemize}

\paragraph{Conclusiones del teorema.} Tenemos que:
\begin{enumerate}
    \item $\equiv_L \; \Rightarrow \; \ca{A}_{\equiv_L}$ produce el autómata con la menor cantidad de estados.
    \item Todo autómata $\ca{A}$ tal que $\equiv_\ca{A} = \equiv_L$ son \textbf{isomorfos} (``equivalentes'').
    \item El \textbf{algoritmo de minimización} produce un autómata isomorfo $\ca{A}_{\equiv_L}$.
\end{enumerate}

\paragraph{Demostración punto 3.} Sea $\ca{A} = (Q,\Sigma,\delta,q_0,F)$ un autómata que acepta $L$ ya \textbf{minimizado}:
$$
    \begin{aligned}
        u \equiv_L v \quad & \Leftrightarrow \quad(u \cdot w \in L \Leftrightarrow v \cdot w \in L) \quad \forall w \in \Sigma^*                                                                                                                \\
                           & \Leftrightarrow \quad\left(\hat{\delta}\left(q_0, u \cdot w\right) \in F \Leftrightarrow \hat{\delta}\left(q_0, v \cdot w\right) \in F\right) \quad \forall w \in \Sigma^*                                         \\
                           & \Leftrightarrow \quad\left(\hat{\delta}\left(\hat{\delta}\left(q_0, u\right), w\right) \in F \Leftrightarrow \hat{\delta}\left(\hat{\delta}\left(q_0, v\right), w\right) \in F\right) \quad \forall w \in \Sigma^* \\
                           & \Leftrightarrow \quad \hat{\delta}\left(q_0, u\right) \approx_\mathcal{A} \hat{\delta}\left(q_0, v\right)                                                                                                          \\
                           & \Leftrightarrow \quad \hat{\delta}\left(q_0, u\right)=\hat{\delta}\left(q_0, v\right)                                                                                                                              \\
                           & \Leftrightarrow \quad u \equiv_\mathcal{A} v
    \end{aligned}
$$
\hfill $\blacksquare$

\subsection{Autómatas en dos direcciones}

\subsubsection{Definición de un 2DFA}

\fig{img/cap2/2dfa.png}{0.2}{Representación de un 2DFA}

\paragraph{Definición.} Un autómata finito determinista en 2 direcciones (2DFA) es una estructura:
\alignformula{
    \mathcal{A}=\left(Q, \Sigma, \vdash, \dashv, \delta, q_0, q_f\right)
}
\begin{itemize}
    \item $Q$ es un conjunto finito de estados.
    \item $\Sigma$ es el alfabeto del input.
    \item $\vdash$ y $\dashv$ son las marcas (símbolos) iniciales y finales.
    \item $\delta: Q \times(\Sigma \cup\{\vdash, \dashv\}) \rightarrow Q \times\{\leftarrow, \rightarrow\}$ es la \textbf{función parcial de transición}.
    \item $q_0$ es el estado inicial.
    \item $q_f$ es el estado final.
\end{itemize}

\ejemplo{}{}{
    \img{img/cap2/ejemplo7.png}{0.225}
}

\paragraph{Configuración.} Sea $\ca{A}$ un 2DFA y $w = a_1 \ldots a_n \in \Sigma^*$ un input. Definimos $a_0 =\ \vdash$ y $a_{n+1} =\ \dashv$ tal que el input se define como:
\alignformula{
    a_0 a_1 \ldots a_n a_{n+1} =\ \vdash \cdot\ w\ \cdot\ \dashv
}
Una \textbf{configuración} de $\ca{A}$ sobre $w$ viene dado por un par:
\alignformula{
    (q, i) \in Q \times\{0, \ldots, n+1\}
}
\begin{itemize}
    \item $q$ es el \textbf{estado actual} del autómata.
    \item $i$ es la \textbf{posición actual} de la cabeza lectora.
\end{itemize}

Se define la relación de \textbf{siguiente configuración} $\stackrel{\mathcal{A}}{\longmapsto}$ de $\ca{A}$ sobre $w$ como:
\alignformula{
    (p, i) \stackrel{\mathcal{A}}{\longmapsto}(q, j)
}
tal que:
\begin{itemize}
    \item Si $\delta(p, a_i) = (q, \rightarrow)$, entonces $(p, i) \stackrel{\mathcal{A}}{\longmapsto}(q, i+1)$.
    \item Si $\delta\left(p, a_i\right)=(q, \leftarrow)$, entonces $(p, i) \stackrel{\mathcal{A}}{\longmapsto}(q, i-1)$.
\end{itemize}

\paragraph{Ejecución.} Una ejecución (o \textit{run}) $\rho$ de $\ca{A}$ sobre $w$ es una secuencia de configuraciones:
\alignformula{
    \rho:\left(p_0, i_0\right) \rightarrow\left(p_1, i_1\right) \rightarrow \ldots \rightarrow\left(p_m, i_m\right)
}
donde $p_0 = q_0$ y $i_0 = 0$. Además, $\left(p_j, i_j\right) \stackrel{\mathcal{A}}{\longmapsto}\left(p_{j+1}, i_{j+1}\right) \quad \forall j \in[0, m-1]$. \medbreak

Una ejecución $\rho$ de $\ca{A}$ sobre $w$ es de \textbf{aceptación} si:
\alignformula{
    p_m=q_f \quad y \quad i_m=n+1
}

\paragraph{Aceptación.} Decimos que $\ca{A}$ \textbf{acepta} $w$ si hay una ejecución de $\ca{A}$ sobre $w$ que es de \textbf{aceptación}. Así, el \textbf{lenguaje aceptado} por $\ca{A}$ se define como:
\alignformula{
    \mathcal{L}(\mathcal{A})=\left\{w \in \Sigma^* \mid \mathcal{A} \text { acepta } w\right\}
}

\textbf{Ojo:} Notemos que un 2DFA puede pasar por error o NO parar nunca.

\subsubsection{2DFA vs DFA}

Para todo lenguaje regular $L$ existe un 2DFA $\ca{A}$:
\alignformula{
    L = \ca{L}(\ca{A})
}
En otras palabras, DFA $\subseteq$ 2DFA. \bigbreak

¿Son los 2DFA más poderosos que los DFA? En esta sección, demostraremos que no, es decir, pueden representar al mismo conjunto de lenguajes.

\paragraph{¿Cuánta información puede almacenar un 2DFA?} Veamos la siguiente figura:
\begin{multicols}{2}
    \img{img/cap2/2dfa_1.png}{0.2}

    \textit{``Cada vez que $\ca{A}$ cruce de $w$ a $u$ en el estado $p$, $\ca{A}$ cruzará de regreso en el estado $q$''}. \medbreak

    Este comportamiento solo depende de $u$, y no de $w$:
    \img{img/cap2/2dfa_2.png}{0.175}
\end{multicols}

Para cada $u \in \Sigma^*$, definimos la función $T_u: Q \cup\{\bullet\} \rightarrow Q \cup\{\perp\}$ tal que:
\begin{itemize}
    \item $T_u(p) = q \quad$ si, y sólo si, $\quad$ desde $(p, |u|)$ cruza en la configuración $(q, |u|+1)$.
    \item $T_u(p) = \perp \quad$ si, y sólo si, $\quad$ desde $(p, |u|)$ nunca cruza de $u$.
    \item $T_u(\bullet) = q \quad$ si, y sólo si, $\quad$ desde $(q_0,0)$ cruza por primera vez con $(q, |u| + 1)$.
    \item $T_u(\bullet) = q \quad$ si, y sólo si, $\quad$ desde $(q_0,0)$ cruza por primera vez con $(q, |u| + 1)$.
    \item $T_u(\bullet) = \perp \quad$ si, y sólo si, $\quad$ desde $(q_0,0)$ nunca cruza de $u$.
\end{itemize}

\begin{multicols}{2}
Suponga que ahora tenemos una palabra $v$ tal que:
$$
    T_v = T_u
$$
\img{img/cap2/2dfa_3.png}{0.2}

Entonces, $v$ es \textbf{indistingible} de $u$ según $\ca{A}$. En otras palabras:
$$
    u \cdot w \in \mathcal{L}(\mathcal{A}) \Leftrightarrow v \cdot w \in \mathcal{L}(\mathcal{A}) \quad \text { para todo } w \in \Sigma^*
$$
Lo anterior tiene conexión con las relaciones de Myhill-Nerode, que vimos en una sección anterior.
\end{multicols}
Luego, respecto a la función $T_u$ podemos decir que:
\begin{enumerate}
    \item Definimos la relación $\equiv_T$ entre palabras en $\Sigma^*$ tal que:
    $$
    u \equiv_T v \text { si, y solo si, } \quad T_u=T_v
    $$
    es una \textbf{relación de equivalencia}.
    \item $\equiv_T$ es una \textbf{congruencia por la derecha}:
    $$
    u \equiv_T v \Rightarrow \forall w \cdot u \cdot w \equiv_T v \cdot w
    $$
    \item $\equiv_T$ \textbf{refina} a $\ca{L}(\ca{A})$:
    $$
    u \equiv T v \Rightarrow(u \in L \Leftrightarrow v \in L)
    $$
    \item La relación $\equiv_T$ tiene una \textbf{cantidad finita} de clases de equivalencia:
    $$
    T: Q \cup\{\bullet\} \rightarrow Q \cup\{\perp\}
    $$
    Así, $\equiv_{\ca{L}(\ca{A})}$ también tiene una cantidad \textbf{finita} de \textbf{clases de equivalencia}.
\end{enumerate}

\teorema{}{}{
    Para todo 2DFA $\ca{A}$ existe un DFA $\ca{A}'$ tal que:
    $$
    \ca{L}(\ca{A}) = \ca{L}(\ca{A}')
    $$
    En otras palabras, 2DFA $\equiv$ DFA.
}

El DFA del teorema anterior se construye usando el Teorema de Myhill-Nerode a partir de las funciones $T_u$. \medbreak

La demostración del teorema y la construcción queda como ejercicio propuesto para el lector.
% \section{Algoritmos para lenguajes regulares}

\subsection{Evaluación de expresiones regulares}

\paragraph{Expresiones regulares en la práctica.} El lenguaje para definir expresiones regulares en la práctica se conoce como \textbf{RegEx} (o RexExp). Las sintaxis más usadas para definir \textbf{RegEx} son:
\begin{enumerate}
    \item POSIX (Portable Operating System Interface for uniX).
    \item Perl (PCRE $=$ Perl Compatible Regular Expressions).
\end{enumerate}

\begin{table}[H]
    \centering
    \begin{tabular}{lcc}
                          & RegEx                                     & Teoría                                                       \\
        \hline
        Carácter          & $\texttt{a}$                              & $\mathrm{a}$                                                 \\
        Escape            & $\backslash+$                             & $-$                                                          \\
        Cualquiera        & $\cdot$                                   & $\Sigma$                                                     \\
        Clase             & {$\texttt{[abc]}$}                        & $(a+b+c)$                                                    \\
        Clase consecutivo & {$\texttt{[a-zA-Z]}$}                     & $(a+\cdots+z+A+\cdots+Z)$                                    \\
        Clase exclusivo   & {$\texttt{[} \texttt{\^{}}\texttt{0-9]}$} & $\left(++_{a \in \Sigma-\{0, \ldots, 9\}} \mathrm{a}\right)$ \\
        Alternación       & $\texttt{cat} \mid \texttt{dog}$          & $\mathrm{cat}+\mathrm{dog}$                                  \\
        0 o 1             & $\texttt{R} ?$                            & $R^?$                                                        \\
        1 o más           & $\texttt{R}+$                             & $R^{+}$                                                      \\
        0 o más           & $\texttt{R} *$                            & $R^*$                                                        \\
        entre $n$ y $m$   & $\texttt{R}\{\texttt{n}, \texttt{m}\}$    & $R^n(R+\epsilon)^{m-n}$                                      \\
        Backreference     & $(\mathrm{R}) \backslash 1$               & $?$                                                          \\
        $\ldots$          & $\cdots$                                  & $\cdots$
    \end{tabular}
\end{table}

\paragraph{¿Cómo evaluamos una expresión regular?} Tenemos el siguiente problema:
\begin{table}[H]
    \centering
    \begin{tabular}{ll}
        $\texttt{PROBLEMA:}$ & Evaluación de expresiones regulares              \\
        $\texttt{INPUT:}$    & una expresión regular $R$                        \\
                             & un documento $w$                                 \\
        $\texttt{OUTPUT:}$   & $\texttt{TRUE}$ si, y sólo si, $w \in \ca{L}(R)$
    \end{tabular}
\end{table}

Donde el tamaño del input está dado por:
\begin{itemize}
    \item $|R|:=$ número de letras y operadores.
    \item $|w|:=$ largo del documento.
\end{itemize}

Buscamos un algoritmo polinomial en $\lvert R \rvert$ y $\lvert w\rvert$. Algo a notar es que $\lvert R\rvert <{}<\lvert w \rvert$:
\begin{itemize}
    \item $|R|$ puede ser del orden \textbf{decenas} operadores ($\sim$ 1 KB).
    \item $|w|$ puede ser del orden de \textbf{miles a millones} de símbolos ($\sim$ 100MB).
\end{itemize}

\paragraph{Análisis de tiempo diferenciado.} Tenemos dos tipos:
\begin{itemize}
    \item \textbf{Combined complexity:} expresión y documento son parte del input.
    \item \textbf{Data complexity:} solo documento es parte del input (tamaño de expresión es considerada una constante).
\end{itemize}

Buscamos algoritmos que sean \textbf{lineales en data complexity} y ojalá \textbf{polinomiales en combined complexity}. \medbreak

Volviendo a nuestro problema:
\begin{table}[H]
    \centering
    \begin{tabular}{ll}
        $\texttt{PROBLEMA:}$ & Evaluación de expresiones regulares              \\
        $\texttt{INPUT:}$    & una expresión regular $R$                        \\
                             & un documento $w$                                 \\
        $\texttt{OUTPUT:}$   & $\texttt{TRUE}$ si, y sólo si, $w \in \ca{L}(R)$
    \end{tabular}
\end{table}

Los pasos que podemos seguir para resolverlo son:
\begin{enumerate}
    \item Convertimos $R$ a un $\epsilon$-NFA $\ca{A}_R$.
    \item Verificamos si $w \in \ca{L}(\ca{A}_R)$.
\end{enumerate}
Ahora, el tamaño del input es:
\begin{itemize}
    \item $|w|:=$ largo del documento.
    \item $|\ca{A}|:= |Q| + |\Delta|$.
\end{itemize}

Hay varias soluciones para resolver nuestro problema, que iremos enumerado a continuación.

\paragraph{Solución 1: Backtracking.} Esta solución es la que usa la mayoría de los motores de RegEx en la práctica. Se puede hacer una prueba en \url{https://regexr.com/}.

\paragraph{Solución 2: DFA.} Para un DFA $\ca{A} = (Q,\Sigma,\delta, q_0, F)$ y una palabra $w = a_1 \ldots a_n$.
\vspace{-8pt}
\begin{algorithm}[hbt!]
    % \caption*{An algorithm with caption}\label{alg:two}
    \setstretch{1.25}
    \DontPrintSemicolon
    \SetKwFunction{FevalDFA}{eval-DFA}
    \SetKwProg{Fn}{Function}{:}{}
    \SetKw{Check}{check}
    \Fn{\FevalDFA{$\ca{A}$, $w$}}{
        $q := q_0$

        \For{$i=1$ \KwTo $n$}{
            $q := \delta(q,a_i)$
        }

        \Return{\Check $(q \in F)$}
    }
\end{algorithm}
\vspace{-8pt}
\paragraph{Solución 3: NFA determinización.} Para un NFA $\ca{A} = (Q,\Sigma,\Delta, I, F)$ y una palabra $w = a_1 \ldots a_n$.
\vspace{-8pt}
\begin{algorithm}[hbt!]
    % \caption*{An algorithm with caption}\label{alg:two}
    \setstretch{1.25}
    \DontPrintSemicolon
    \SetKwFunction{FevalNFA}{eval-NFA}
    \SetKwFunction{FevalDFA}{eval-DFA}
    \SetKwProg{Fn}{Function}{:}{}
    \Fn{\FevalNFA{$\ca{A}$, $w$}}{
    $\ca{A}^{\text{det}} := \texttt{NFAtoDFA}(\ca{A})$

    \Return{\FevalDFA{$\ca{A}^{\text{det}},w$}}
    }
\end{algorithm}

¿Es necesario construir la determinización completa? La respuesta es que no. Recordemos que es la determinización. Para un autómata no-determinista $\ca{A} = (Q,\Sigma,\Delta,I,F)$, definimos el autómata determinista (\textbf{determinización} de $\ca{A}$):
\alignformula{
\ca{A}^{\text{det}} = (2^Q,\Sigma,\delta^{\text{det}},q_0^\text{det},F^\text{det})
}
\begin{itemize}
    \item $2^Q = \{S \mid S\subseteq Q\}$ es el conjunto potencia de $Q$.
    \item $q_0^\text{det} = I$.
    \item $\delta^\text{det}: 2^Q \times \Sigma \to 2^Q$ tal que:
          $$
              \delta^\text{det}(S,a) = \{q \in Q \mid \exists p \in S.\ (p,a,q) \in \Delta\}
          $$
    \item $F^\text{det} = \{S \in 2^Q \mid S \cap F \neq \varnothing\}$, es decir, todos los conjuntos que tengan al menos un estado final.
\end{itemize}

\paragraph{Solución 4: NFA \textit{on-the-fly}.} Fijémonos en la función de transición $\delta^\text{det}: 2^Q \times \Sigma \to 2^Q$ tal que:
$$
    \delta^\text{det}(S,a) = \{q \in Q \mid \exists p \in S.\ (p,a,q) \in \Delta\}
$$

La estrategia \textit{on-the-fly} corresponde a:
\begin{enumerate}
    \item Mantenemos un conjunto $S$ de estados actuales.
    \item Por cada nueva letra $a$, calculamos el conjunto $\delta^{\text{det}}(S,a)$.
\end{enumerate}
\fig{img/cap3/nfaonthefly.png}{0.2}{Ejemplo de NFa \textit{on-the-fly}}

Definimos el algoritmo para un NFA $\ca{A}=(Q,\Sigma,\Delta,I,F)$ y una palabra $w = a_1 \ldots a_n$ como:
\vspace{-8pt}
\begin{algorithm}[hbt!]
    % \caption*{An algorithm with caption}\label{alg:two}
    \setstretch{1.25}
    \DontPrintSemicolon
    \SetKwFunction{FevalNFAonthefly}{eval-NFAonthefly}
    \SetKwProg{Fn}{Function}{:}{}
    \SetKw{Check}{check}
    \Fn{\FevalNFAonthefly{$\ca{A}$, $w$}}{
        $S := I$

        \For{$i = 1$ \KwTo $n$}{
            $S_\text{old} := S$

            $S := \varnothing$

            \ForEach{$p \in S_\text{old}$}{
                $S := S \cup \{q \mid (p, a_i,q) \in \Delta\}$
            }
        }

        \Return{\Check $(S \cap F \neq \varnothing)$}
    }
\end{algorithm}

\paragraph{Resumen y complejidades.} Para un autómata $\ca{A}$ y una palabra $w = a_1 \ldots a_n$ tenemos que:

\begin{table}[H]
    \centering
    \begin{tabular}{lc}
                                           & Tiempo                                        \\
        \hline
        Backtracking                       & $\mathcal{O}\left(|\mathcal{A}|^{|w|}\right)$ \\
        DFA                                & $\mathcal{O}(|\mathcal{A}|+|w|)$              \\
        NFA                                & $\mathcal{O}\left(2^{|Q|}+|w|\right)$         \\
        $\epsilon$-NFA \textit{on-the-fly} & $\mathcal{O}(|\mathcal{A}| \cdot|w|)$
    \end{tabular}
\end{table}


\subsection{Transductores}
\fig{img/cap3/transductor.png}{0.2}{Representación de un transductor}

\paragraph{Definición.} Un transductor (en inglés, \textit{transducer}) es una tupla:
\alignformula{
    \ca{T} = (Q,\Sigma,\Omega,\Delta,I,F)
}
\begin{itemize}
    \item $Q$ es un conjunto finito de estados.
    \item $\Sigma$ es el alfabeto del input.
    \item $\Omega$ es el alfabeto del \textbf{output}.
    \item $\Delta \subseteq Q \times(\Sigma \cup\{\epsilon\}) \times(\Omega \cup\{\epsilon\}) \times Q$ es la \textbf{relación de transición}.
    \item $I \subseteq Q$ es un conjunto de estados iniciales.
    \item $F \subseteq Q$ es el conjunto de estados finales.
\end{itemize}

\ejemplo{}{}{
    Algunos transductores:
    \begin{figure}[H]
        \centering
        \includegraphics[scale=0.35]{img/cap3/ejemplo1_1.png}
        \includegraphics[scale=0.35]{img/cap3/ejemplo1_2.png}
    \end{figure}
}

\paragraph{Configuración.} Sea $\ca{T}$ un transductor. Definimos:
\begin{itemize}
    \item Un par $(q, u, v) \in Q \times \Sigma^* \times \Omega^*$ es una \textbf{configuración} de $\ca{T}$.
    \item Una configuración $(q,u,\epsilon)$ es \textbf{inicial} si $q \in F$.
    \item Una configuración $(q,\epsilon,v)$ es \textbf{final} si $q \in F$.
\end{itemize}

\textit{``Intuitivamente, una configuración $(q, au, vb)$ representa que $\ca{T}$ se encuentra en el estado $q$ procesando la palabra $au$ y leyendo $a$, y hasta ahora grabó la palabra $vb$ y el último símbolo impreso es $b$''.}

\paragraph{Ejecución.} Se define la relación $\vdash_\ca{T}\; \subseteq\left(Q \times \Sigma^* \times \Omega^*\right) \times\left(Q \times \Sigma^* \times \Omega^*\right)$ de \textbf{siguiente-paso} entre configuraciones de $\ca{T}$:
\alignformula{
    \left(p, u_1, v_1\right) \vdash_{\mathcal{T}}\left(q, u_2, v_2\right)
}
si, y sólo si, existe $(p,a,b,q) \in \Delta$ tal que $u_1 = a \cdot u_2$ y $v_2 = v_1 \cdot b$. \medbreak

Se define $\vdash_\ca{T}^*$ como la clausura \textbf{refleja} y \textbf{transitiva} de $\vdash_\ca{T}$:
\alignformula{
    \text{para toda configuración } (q,u,v): \quad &(q, u, v) \vdash_{\mathcal{T}}^*(q, u, v) \\
    \text{si } \left(q_1, u_1, v_1\right) \vdash_{\mathcal{T}}^*\left(q_2, u_2, v_2\right) \text{ y } \left(q_2, u_2, v_2\right) \vdash_{\mathcal{T}}\left(q_3, u_3, v_3\right): \quad &\left(q_1, u_1, v_1\right) \vdash_{\mathcal{T}}^*\left(q_3, u_3, v_3\right)
}
\ejemplo{}{}{
    \vspace{-10pt}
    \img{img/cap3/ejemplo2.png}{0.3}
}

\paragraph{Función definida por un transductor.} Dado un transductor $\ca{T}$, decimos que:
\begin{itemize}
    \item $\ca{T}$ \textbf{entrega} $v$ con \textbf{input} $u$ si existe una configuración inicial $(q_0, u, \epsilon)$ y una configuración final $(q_f, \epsilon, v)$ tal que:
          \alignformula{
              \left(q_0, u, \epsilon\right) \stackrel{\vdash_{\mathcal{T}}^*}{ }\left(q_f, \epsilon, v\right)
          }
    \item Se define la función $\llbracket \mathcal{T} \rrbracket: \Sigma^* \rightarrow 2^{\Omega^*}$:
          \alignformula{
              \llbracket \mathcal{T} \rrbracket(u)=\left\{v \in \Omega^* \mid \mathcal{T} \text { entrega } v \text { con input } u\right\}
          }
    \item Se dice que $f: \Sigma^* \rightarrow 2^{\Omega^*}$ es una \textbf{función racional} si existe un transductor $\ca{T}$ tal que $f=\llbracket \mathcal{T} \rrbracket$.
    \item Un transductor define una función de palabras a conjuntos de palabras.
\end{itemize}

\paragraph{Dos interpretaciones para un transductor.} Podemos ver a $\llbracket \mathcal{T} \rrbracket$ de dos formas:
\begin{enumerate}
    \item $\ca{T}$ define la \textbf{función} $\llbracket \mathcal{T} \rrbracket: \Sigma^* \rightarrow 2^{\Omega^*}$:
          $$
              \llbracket \mathcal{T} \rrbracket(u)=\left\{v \in \Omega^* \mid \mathcal{T} \text { entrega } v \text { con input } u\right\}
          $$
    \item $\ca{T}$ define la \textbf{relación} $\left[\mathcal{T} \rrbracket \subseteq \Sigma^* \times \Omega^*\right.$:
          $$
              (u, v) \in \llbracket \mathcal{T} \rrbracket \quad \text { si, y solo si, } \quad \mathcal{T} \text { entrega } v \text { con input } u
          $$
\end{enumerate}

Desde ahora, hablaremos de función o relación \textbf{indistintamente} y hablaremos de las \textbf{relaciones racionales} (definidas por un transductor).

\paragraph{Lenguaje de input y ouput.} Para una relación $R \subseteq \Sigma^* \times \Omega^*$ se define:
\begin{itemize}
    \item $\pi_1(R)=\left\{u \in \Sigma^* \mid \exists v \in \Omega^* .\ (u, v) \in R\right\}$.
    \item $\pi_2(R)=\left\{v \in \Omega^* \mid \exists u \in \Sigma^* .\ (u, v) \in R\right\}$.
\end{itemize}

\teorema{}{}{
    Si $\ca{T}$ es un transductor, entonces $\pi_1(\llbracket \mathcal{T} \rrbracket)$ y $\pi_2(\llbracket \mathcal{T} \rrbracket)$ son lenguajes regulares sobre $\Sigma$ y $\Omega$, respectivamente.
}

\paragraph{Idea demostración teorema 12.} Para $\ca{T} = (Q,\Sigma,\Omega,\Delta,I,F)$, defina $\ca{A}_1 = (Q, \Sigma, \Delta_1, I, F)$ tal que
$$
    (p, a, q) \in \Delta_1 \quad \text { si, y solo si, } \quad \exists b \in \Omega \cup\{\epsilon\} .\ (p, a, b, q) \in \Delta
$$
y demuestre que $\ca{L}(\ca{A}_1) = \pi_1(\llbracket \mathcal{T} \rrbracket)$.

\teorema{}{}{
    Sea $\ca{T}_1$ y $\ca{T}_2$ dos transductores con $\Sigma$ y $\Omega$ alfabetos de input y output. Las siguientes son \textbf{relaciones racionales}:
    \begin{enumerate}
        \item $$
                  \llbracket \mathcal{T}_1 \rrbracket \cup \llbracket \mathcal{T}_2 \rrbracket=\left\{(u, v) \in \Sigma^* \times \Omega^* \mid(u, v) \in \llbracket \mathcal{T}_1 \rrbracket \vee(u, v) \in \llbracket \mathcal{T}_2 \rrbracket\right\}
              $$
        \item $\llbracket \mathcal{T}_1 \rrbracket \cdot \llbracket \mathcal{T}_2 \rrbracket=\left\{\left(u_1 u_2, v_1 v_2\right) \in \Sigma^* \times \Omega^* \mid\left(u_1, v_1\right) \in \llbracket \mathcal{T}_1 \rrbracket \wedge\left(u_2, v_2\right) \in \llbracket \mathcal{T}_2 \rrbracket\right\}$
        \item $\llbracket \mathcal{T}_1 \rrbracket^*=\bigcup_{k=0}^{\infty} \llbracket \mathcal{T}_1 \rrbracket^k$
    \end{enumerate}
}

La demostración de este teorema queda como ejercicio propuesto al lector.

\teorema{}{}{
    Existen transductores $\ca{T}_1$ y $\ca{T}_2$ sobre $\Sigma$ y $\Omega$ tal que:
    $$
        \llbracket \mathcal{T}_1 \rrbracket \cap \llbracket \mathcal{T}_2 \rrbracket=\left\{(u, v) \in \Sigma^* \times \Omega^* \mid(u, v) \in \llbracket \mathcal{T}_1 \rrbracket \wedge(u, v) \in \llbracket \mathcal{T}_2 \rrbracket\right\}
    $$
    NO es una relación racional.
}
\paragraph{Demostración teorema 14.} Considere los siguientes transductores:
\img{img/cap3/teo14.png}{0.2}
Vemos que $\llbracket \mathcal{T}_1 \rrbracket \cap \llbracket \mathcal{T}_2 \rrbracket=\left\{\left(a^n, b^n c^n\right) \mid n \geq 0\right\}$, pero el lenguaje que define el ouput no es regular, y por lo tanto $\llbracket \mathcal{T}_1 \rrbracket \cap \llbracket \mathcal{T}_2 \rrbracket$ no es racional. \hfill $\blacksquare$

\paragraph{Definición.} Decimos que un transductor $\ca{T}$ define una \textbf{función parcial} si:
\alignformula{
    \text { para todo } u \in \Sigma^* \text { se tiene que }|\llbracket \mathcal{T} \rrbracket(u)| \leq 1 \text {. }
}

\paragraph{Definición.} Decimos que $\ca{T}=(Q,\Sigma,\Omega,\Delta,I,F)$ es \textbf{determinista} si cumple que:
\begin{enumerate}
    \item $\ca{T}$ define una función $\llbracket \mathcal{T} \rrbracket: \Sigma^* \rightarrow \Omega^*$.
    \item para todo $(p,a_1,b_1,q_1)\in\Delta$ y $(p,a_2,b_2,q_2)\in \Delta$, si $a_1=a_2$, entonces
          $$
              b_1=b_2 \quad\text{ y }\quad q_1=q_2
          $$
          Es decir, si tenemos dos transiciones desde un estado $p$ a otros dos estados $q_1$ y $q_2$, y ambas transiciones leen lo mismo: $a_1=a_2$, entonces lo que imprime el transductor y los estados deben ser iguales: $b_1=b_2$ y $q_1=q_2$. Básicamente, buscamos que sean la misma transición.
    \item si $(p,\epsilon,b,q) \in \Delta$, entonces para todo $(p,a',b',q')\in\Delta$, se tiene que
          $$
              (a',b',c')=(\epsilon,b,q)
          $$
          Es decir, si tenemos una $\epsilon$-transición, entonces es la única transición que sale desde $p$.
\end{enumerate}
\ejemplo{}{}{
    \img{img/cap3/ejemplo3.png}{0.35}
}

\subsection{Análisis léxico}

PENDIENTE.

\subsection{Algoritmo de Knuth-Morris-Prat}

\paragraph{Pattern matching.} Veamos el siguiente problema. Dado un \textbf{patrón} $w = w_1 \ldots w_m$ y un documento $d = d_1 \ldots d_n$, encontrar todas las posiciones donde aparece $w$ en $d$, o sea, enumerar:
$$
    \left\{(i, j) \mid w=d_i d_{i+1} \ldots d_j\right\}
$$

Podríamos implementar la siguiente solución ingenua (y poco eficiente):
\vspace{-8pt}
\begin{algorithm}[hbt!]
    % \caption*{An algorithm with caption}\label{alg:two}
    \setstretch{1.25}
    \DontPrintSemicolon
    \SetKw{Output}{output}
    \For{$i=0$ \KwTo $n-m$}{
        $j := 1$

        \While{$j \leq m \wedge w_j = d_{i+j}$}{
            $j := j + 1$
        }

        \If{$j > m$}{
            \Output $(i+1, i+m-1)$
        }
    }
\end{algorithm}
\vspace{-8pt}
\subsubsection{Autómata de un patrón}

\paragraph{Definición.} Dado una palabra $w = w_1 \ldots w_m$, sea el NFA $\ca{A}_w = (Q,\Sigma,\Delta,I, F)$ tal que:
\begin{itemize}
    \item $Q = \{0,1,\ldots,m\}$
    \item $\Delta=\{(0, a, 0) \mid a \in \Sigma\} \cup\left\{\left(i, w_{i+1}, i+1\right) \mid i<m\right\}$
    \item $I = \{0\}$ y $F = \{m\}$.
\end{itemize}
\ejemplo{Palabra $w =$ nano}{}{
    \vspace{-8pt}
    \img{img/cap3/ejemplo4.png}{0.325}
}

Podemos usar $\ca{A}_w$ para encontrar todas las apariciones de $w$ en $w$ haciendo su \textbf{determinización}. \medbreak

\paragraph{Definción.} Sea $\mathcal{A}_w^{\operatorname{det}}=\left(Q^{\operatorname{det}}, \Sigma, \delta^{\operatorname{det}},\{0\}, F^{\operatorname{det}}\right)$ la determinización de $\ca{A}_w$ tal que $Q^\text{det}$ contiene \textbf{solo los estados alcanzables} desde $\{0\}$.

\ejemplo{Palabra $w =$ nano y $d =$ un nano no nana}{}{\
    \vspace{-8pt}
    \img{img/cap3/ejemplo5.png}{0.325}
    $$
        \begin{array}{ccccccccccccccccccc}
            q_0 & q_0 & q_1 & q_0 & q_1 & q_2 & q_3 \checkmark & q_4 & q_0 & q_1 & q_0 & q_0 & q_1 & q_2 & q_3 & q_2 \\
            u   & n   &     & n   & a   & n   & o              &     & n   & o   &     & n   & a   & n   & a         \\
            1   & 2   & 3   & 4   & 5   & 6   & 7              & 8   & 9   & 10  & 11  & 12  & 13  & 14  & 15
        \end{array}
    $$
}

\teorema{}{}{
    Para todo $S \in Q^\text{det}$ y $i \in \{0,1,\ldots,m\}$ se cumple que
    $$
        i \in S \quad \text { si, y solo si, } \quad w_1 \ldots w_i \text { es un sufijo de } w_1 \ldots w_{\max (S)} \text {. }
    $$
}

\paragraph{Corolarios.} Tenemos que:
\begin{itemize}
    \item Para todo $S_1,S_2 \in Q^\text{det}$, si $\max(S_1) = \max(S_2)$, entonces $S_1 = S_2$.
    \item $\ca{A}_w^\text{det}$ tiene $|w| + 1$ estados y a lo más $\ca{O}(|w|^2)$ transiciones.
\end{itemize}

Por lo tanto, encontrar todos los substrings de $w$ en $d$ toma tiempo $\ca{O}(|d| + |w|^2)$.

\paragraph{Demostración teorema 15.} Sea $S \in Q^\text{det}$ un conjunto de estados cualquiera alcanzable desde $\{0\}$. Entonces, existe una palabra $u = a_1 \ldots a_k$ tal que $\hat{\delta}^{\operatorname{det}}(\{0\}, u)=S$. Por la demostración de que $\mathcal{L}\left(\mathcal{A}^{\mathrm{det}}\right)=\mathcal{L}(\mathcal{A})$ para todo NFA $\ca{A}$, sabemos que $j \in S$ si, y sólo si, existe una ejecución $\ca{A}_w$ sobre $u$:
$$
    0=q_0 \stackrel{a_1}{\rightarrow} q_1 \stackrel{a_2}{\rightarrow} \ldots \stackrel{a_k}{\rightarrow} q_k=j
$$
Por la definición de $\ca{A}_w$ esta ejecución es de la forma:
$$
    0 \stackrel{a_1}{\rightarrow} 0 \stackrel{a_2}{\rightarrow} \ldots \stackrel{a_{k-j}}{\rightarrow} 0 \underbrace{\stackrel{a_{k-j+1}}{\rightarrow} 1 \stackrel{a_{k-j+2}}{\rightarrow} 2 \ldots \stackrel{a_k}{\rightarrow}}_{w_1 \ldots w_j} j
$$
Por lo tanto, $w_1 w_2 \ldots w_j$ es sufijo de $a_1 \ldots a_k$. Usaremos este último hecho para demostrar \textbf{ambas} direcciones del teorema, que formalizamos a continuación. \medbreak

\textit{Propiedad.} Para toda $u = a_1\ldots a_k$ tal que $\hat{\delta}^{\operatorname{det}}(\{0\}, u)=S$, y para todo $j \leq m$:
$$
    j \in S \quad \text { si, y solo si, } \quad w_1 \ldots w_j \text { es sufijo de } a_1 \ldots a_k
$$

\textit{Demostración $(\Rightarrow)$.} Como $S$ es alcanzable desde $\{0\}$, entonces existe $u = a_1 \ldots a_k$ tal que $\hat{\delta}^{\operatorname{det}}(\{0\}, u)=S$. Como $\max(S) \in S$, entonces $W_1 \ldots W_{\max }(S)$ es sufijo de $a_1 \ldots a_k$. \medbreak

Suponga que $i \in S$. Entonces $w_1 \ldots w_i$ es sufijo de $a_1 \ldots a_k$. Como $i \leq \max(S)$, entonces:
$$
    a_1 a_2 \ldots a_{k-\max(S)} \overbrace{a_{k-\max(S)+1} \ldots a_{k-i} \underbrace{a_{k-i+1}\ldots a_k}_{w_1\ldots w_i}}^{w_1\ldots w_{\max(S)}}
$$
Por lo tanto, $w_1 \ldots w_i$ es sufijo de $w_1 \ldots w_{\max(S)}$. \medbreak

\textit{Demostración $(\Leftarrow)$.} Como $S$ es alcanzable desde $\{0\}$, entonces existe $u = a_1 \ldots a_k$ tal que $\hat{\delta}^{\mathrm{det}}(\{0\}, u)=S$. Como $\max(S) \in S$, entonces $w_1 \ldots w_{\max(S)}$ es sufijo de $a_1 \ldots a_k$. \medbreak

Suponga que $w_1 \ldots w_i$ es sufijo de $w_1 \ldots w_{\max(S)}$. Como $w_1 \ldots w_i$ es sufijo de $w_1 \ldots w_{\max(S)}$ y $w_1 \ldots w_{\max(S)}$ es sufijo de $u$, entonces $w_1 \ldots w_i$ es sufijo de $u = a_1 \ldots a_k$. \medbreak

Por la ``propiedad'', concluimos que $i \in S$. \hfill $\blacksquare$

\subsubsection[Autómata finito con k-lookahead]{Autómata finito con $k$-lookahead}

\paragraph{Definición}. Se definen los siguientes conjuntos de palabras sobre un alfabeto finito $\Sigma$:
\begin{itemize}
    \item $\Sigma_{\bullet}=\Sigma^* \times \Sigma^*$
    \item $\Sigma_{\bullet}^k=\left\{(u, v) \in \Sigma_{\bullet}\mid |uv| = k\right\}$
\end{itemize}

\paragraph{Notación.} En vez de $(u,v) \in \Sigma_\bullet$, escribiremos $u.v \in \Sigma_\bullet$.

\ejemplo{}{}{
    Si $\Sigma = \{a,b\}$, entonces:
    \begin{itemize}
        \item $ab.ba \in \Sigma_\bullet$ y $.aba \in \Sigma_\bullet$
        \item $ab.ba \in \Sigma_\bullet^4$ y $.aba \in \Sigma_\bullet^3$
    \end{itemize}
}

\paragraph{Definición.} Un autómata finito determinista con $k$-lookahead es una tupla:
\alignformula{
    \ca{A} = (Q, \Sigma, \delta, q_0, F)
}
\begin{itemize}
    \item $Q$ es un conjunto finito de estados.
    \item $\Sigma$ es el alfabeto del input.
    \item $q_0$ es el estado inicial.
    \item $F \subseteq Q$ es el conjunto de estados iniciales.
    \item $\delta: Q \times(\Sigma \cup\{\mathbb{\$}\})_{\bullet}^k \rightarrow Q$ es una función parcial, tal que:
          $$
              \text { para todo } p \in Q \text { y } w \in(\Sigma \cup\{\$\})^k:\quad  |\{u . v \mid \delta(p, u \cdot v)=q \text { y } u v=w\}| \leq 1
          $$
\end{itemize}
\ejemplo{}{}{
    \begin{figure}[H]
        \centering
        \includegraphics[scale=0.4]{img/cap3/ejemplo6_1.png}
        \includegraphics[scale=0.4]{img/cap3/ejemplo6_2.png}
    \end{figure}
}

\paragraph{Ejecución.} Sea $\ca{A}$ un DFA con $k$-lookahead. Tenemos que:
\begin{itemize}
    \item Un par $(q, w) \in Q \times(\Sigma \cup\{\$\})^*$ es una \textbf{configuración} de $\ca{A}$.
    \item Una configuración $(q_0, w\$^k)$ es \textbf{inicial}.
    \item Una configuración $(q, \$^k)$ es \textbf{final} si $q \in F$.
\end{itemize}

El sufijo $\$^k$ nos sirve para marcar el finald el input (y simplificar la definición de lookahead al leer el final de la palabra). \medbreak

Se define la relación $\vdash_\ca{A}$ de \textbf{siguiente-paso} entre configuraciones de $\ca{A}$:
\alignformula{
    \left(p_1, w_1\right) \vdash_\mathcal{A}\left(p_2, w_2\right)
}
si, y sólo si, $\delta\left(p_1, u . v\right)=p_2$ y existe $w \in \Sigma^*$ tal que $w_1 = uvw$ y $w_2 = vw$. \medbreak

Se define $\vdash_\ca{A}^*$ como la clausura \textbf{refleja} y transitiva de $\vdash_A$:
\alignformula{
    \text{para toda configuración } (p,w): \quad &(p,w) \vdash_\ca{A}^* (p,w) \\
    \text{si } (p_1, w_1) \vdash_\ca{A}^* \text{ y } (p2,w_2) \vdash_\ca{A} (p_3,w_3): \quad &\left(p_1, w_1\right) \vdash_{\mathcal{A}}^*\left(p_3, w_3\right)
}

Decimos que $(p,u) \vdash_\ca{A}^* (q,v)$ si uno puede ir de $(p,u)$ a $(q,v)$ en \textbf{0 o más pasos}.

\paragraph{Aceptación.} Decimos que $\ca{A}$ \textbf{acepta} $w$ si existe una configuración inicial $(q_0, w\$^k)$ y una configuración final $(q_f, \$^k)$ tal que:
$$
    \left(q_0, w \mathbb{S}^k\right) \vdash_{\mathcal{A}}^*\left(q_f, \mathbb{S}^k\right)
$$
El \textbf{lenguaje aceptado} por $\ca{A}$ se define como:
$$
    \mathcal{L}(\mathcal{A})=\left\{w \in \Sigma^* \mid \mathcal{A} \text { acepta } w\right\}
$$
Vemos que son las mismas definiciones para un $\epsilon$-NFA.

\teorema{}{}{
    Para todo DFA con $k$-lookahead $\ca{A}$ se tiene que $\ca{L}(\ca{A})$ es un \textbf{lenguaje regular}.
}
La demostración de este teorema queda como ejercicio propuesto al lector.

\paragraph{Definición.} Llamaremos un \textbf{lazy autómata} a un DFA con $1$-lookahead.

\subsubsection{Algoritmo KMP}

\paragraph{Construcción de un lazy automata.} Sea $w = w_1 \ldots w_m$ y $\mathcal{A}_w^{\mathrm{det}}=\left(Q^{\mathrm{det}}, \Sigma, \delta^{\mathrm{det}},\{0\}, F^{\mathrm{det}}\right)$ la determinización de $\ca{A}_w$.

\paragraph{Definición.} Para $i \in [0,m]$, sea $S_i$ el \textbf{único estado} en $Q^\text{det}$ tal que $i = \max(S_i)$.

\paragraph{Propiedad.} Para todo $a \in \{w_1,\ldots w_m\}$ y $i \in [0,m-1]$:
\begin{enumerate}
    \item $S_i-\{i\} \in Q^{\operatorname{det}}$
    \item $a=w_{i+1}$, entonces $\delta^{\operatorname{det}}\left(S_i, a\right)=S_{i+1}$.
    \item $a \neq w_{i+1}$, entonces $\delta^{\operatorname{det}}\left(S_i, a\right)=\delta^{\operatorname{det}}\left(S_i-\{i\}, a\right)$.
\end{enumerate}
La demostración de esta propiedad queda como ejercicio propuesto al lector. \medbreak

En base a la propiedad anterior, se define el lazy autómata $\mathcal{A}_w^{\text {lazy}}=\left(Q^{\text {det}}, \Sigma, \delta^{\text {lazy }},\{0\}, F^{\text {det }}\right)$ tal que:
\begin{itemize}
    \item para todo $a\neq w_1$: $\delta^{\text {lazy}}(\{0\}, a .)=\{0\}$.
    \item para todo $a \in \{w_1,\ldots, w_m\}$ y $i \in [0,m-1]$:
          \begin{itemize}
              \item si $a = w_{i_1}$, entonces $\delta^{\text {lazy}}\left(S_i, a_{.}\right)=S_{i+1}$
              \item si $a \neq w_{i+1}$ y $i \neq 0$, entonces $\delta^{\text {lazy}}\left(S_i, . a\right)=S_i-\{i\}$.
          \end{itemize}
\end{itemize}
\ejemplo{}{}{
    \img{img/cap3/ejemplo7.png}{0.35}
}

\teorema{}{}{
    Para todo $w$ se cumple que $\mathcal{L}\left(\mathcal{A}_w^{\text {det}}\right)=\mathcal{L}\left(\mathcal{A}_w^{\text {lazy}}\right)$.
}
La demostración queda como ejercicio propuesto para el lector (usando la propiedad).

\paragraph{Complejidad KMP.} Vemos que:
\begin{itemize}
    \item El número de pasos que $\ca{A}_w^\text{lazy}$ \textbf{consume} letras $= |d|$
    \item El número de pasos que $\ca{A}_w^\text{lazy}$ \textbf{retrocede} $\leq |d|$
    \item El número de \textbf{pasos totales} de $\ca{A}_w^\text{lazy}$ $\leq 2 \cdot |d|$
\end{itemize}
Por lo tanto, la cantidad de pasos es \textbf{lineal} en $\ca{O}(\d|)$.

\paragraph{Algoritmo KMP.} Dado una palabra $w$ y un documento $d$:
\begin{enumerate}
    \item Construimos $\ca{A}_w^\text{lazy}$ desde $\ca{A}_w$.
    \item Ejecutamos $\ca{A}_w^\text{lazy}$ sobre $d$.
\end{enumerate}

El paso 1 toma $\ca{O}(|w|)$ y el paso 2 toma $\ca{O}(|d|)$, por lo tanto, el tiempo del algoritmo es $\ca{O}(|w| + |d|)$. \medbreak

Queda como ejercicio para el lector demostrar que construir $\ca{A}_w^\text{lazy}$ toma tiempo $\ca{O}(|w|)$.





% \section{Lenguajes libres de contexto}

\subsection{Gramáticas libres de contexto}

\subsubsection{Gramáticas}

\paragraph{Definición.} Una \textbf{gramática libre de contexto} (CFG) es una tupla:
\alignformula{
    \ca{G} = (V, \Sigma, P, S)
}
\begin{itemize}
    \item $V$ es un conjunto finito de \textbf{variables} o \textbf{no-terminales}.
    \item $\Sigma$ es el alfabeto finito (o \textbf{terminales}) tal que $\Sigma \cap V = \varnothing$.
    \item $P \subseteq V \times(V \cup \Sigma)^*$ es un subconjunto finito de \textbf{reglas} o \textbf{producciones}.
    \item $S \in V$ es la \textbf{variable inicial}.
\end{itemize}

\ejemplo{}{}{
    Considere la gramática $\ca{G} = (V, \Sigma, P, S)$ tal que:
    \begin{itemize}
        \item $V = \{X, Y\}$
        \item $\Sigma = \{a,b\}$
        \item $\{(X, aXb), (X, Y), (Y, \epsilon)\}$
        \item $S = X$
              \begin{align*}
                  \ca{G}: \quad  X & \to aXb      \\
                  X                & \to Y        \\
                  Y                & \to \epsilon
              \end{align*}
    \end{itemize}
}

\paragraph{Notación.} En este texto:
\begin{itemize}
    \item Para las \textbf{variables} en una gramática usaremos letras mayúsculas: $X,Y,Z,A,B,C,\ldots$
    \item Para los \textbf{terminales} en una gramática usaremos letras minúsculas: $a,b,c,\ldots$
    \item Para palabras en $(V \cup \Sigma)^*$ usaremos símbolos: $\alpha, \beta, \gamma, \ldots$
    \item Para una producción $(A,\alpha) \in P$ la escribimos como: $A \to \alpha$
\end{itemize}

\paragraph{Simplificación.} Si tenemos un conjunto de reglas de la forma:
$$
    \begin{array}{lll}
        X & \rightarrow & \alpha_1 \\
        X & \rightarrow & \alpha_2 \\
          & \cdots      &          \\
        X & \rightarrow & \alpha_n
    \end{array}
$$
entonces escribimos estas reglas \textbf{sucintamente} como
$$
    X \to \alpha_1 \mid \alpha_2 \mid \cdots \mid \alpha_n
$$
Recordando que $\alpha_1, \alpha_2, \ldots, \alpha_n \in (V \cup \Sigma)^*$.

\ejemplo{}{}{
    La gramática del ejemplo anterior:
    \begin{align*}
        \ca{G}: \quad  X & \to aXb      \\
        X                & \to Y        \\
        Y                & \to \epsilon
    \end{align*}
    Podemos escribirla en notación \textbf{sucinta} como:
    \begin{align*}
        \ca{G}: \quad X & \to aXb \mid Y \\
        Y               & \to \epsilon
    \end{align*}
}

\paragraph{Producciones.} Sea $\ca{G}$ una CFG. Definimos la relación $\Rightarrow \subseteq(V \cup \Sigma)^* \times(V \cup \Sigma)^*$ de \textbf{producción} tal que:
\alignformula{
    \alpha \cdot X \cdot \beta \Rightarrow \alpha \cdot \gamma \cdot \beta \quad \text { si, y solo si, } \quad(X \rightarrow \gamma) \in P
}
para todo $X \in V$ y $\alpha, \beta, \gamma \in(V \cup \Sigma)^*$. \medbreak

Si $\alpha X \beta \Rightarrow \alpha \gamma \beta$ entonces decimos que:
\begin{itemize}
    \item $\alpha X \beta$ \textbf{produce} $\alpha \gamma \beta$ o
    \item $\alpha \gamma \beta$ \textbf{es producible} desde $\alpha X \beta$.
    \item $\alpha X \beta \Rightarrow \alpha \gamma \beta$ es \textbf{reemplazar} $\gamma$ en $X$ en la palabra $\alpha X \beta$.
\end{itemize}

\paragraph{Derivaciones.} Sea $\ca{G}$ una CFG. Dadas dos palabras $\alpha, \beta \in(V \cup \Sigma)^*$ decimos que $\alpha$ \textbf{deriva} $\beta$:
\alignformula{
    \alpha \overunder{\Rightarrow}{}{*} \beta
}
si existe $\alpha_1, \alpha_2, \ldots, \alpha_n \in(V \cup \Sigma)^*$ tal que: $\alpha \Rightarrow \alpha_1 \Rightarrow \alpha_2 \Rightarrow \ldots \Rightarrow \beta$, con $\overunder{\Rightarrow}{}{*}$ la \textbf{clausura refleja y transitiva} de $\Rightarrow$, esto es:
\begin{enumerate}
    \item $\alpha \overunder{\Rightarrow}{}{*} \alpha$
    \item $\alpha \overunder{\Rightarrow}{}{*} \beta$ si, y sólo si, existe $\gamma$ tal que $\alpha \overunder{\Rightarrow}{}{*} \gamma$ y $\gamma \Rightarrow \beta$.
\end{enumerate}
para todo $\alpha, \beta \in (V \cup \Sigma)^*$. Notemos que $\Rightarrow$ y $\overunder{\Rightarrow}{}{*}$ son relaciones entre palabras en $(V \cup \Sigma)^*$.

\paragraph{Lenguaje.} Sea $\ca{G}$ una CFG. El \textbf{lenguaje} de una gramática $\ca{G}$ se define como:
\alignformula{
    \mathcal{L}(\mathcal{G})=\left\{w \in \Sigma^* \mid S \stackrel{\star}{\Rightarrow} w\right\}
}

$\ca{L}(\ca{G})$ son todas las palabras en $\Sigma^*$ que se pueden derivar desde $S$.

\ejemplo{}{}{
    Sea $\ca{G}$ una CFG tal que:
    \begin{align*}
        \ca{G}: \quad X & \to aXb \mid Y \\
        Y               & \to \epsilon
    \end{align*}
    \begin{itemize}
        \item Como $X \overunder{\Rightarrow}{}{*} aaabbb$, entonces $aaabbb \in \ca{L}(\ca{G})$.
        \item En general, uno puede demostrar por \textbf{inducción} que:
              $$
                  \mathcal{L}(\mathcal{G})=\left\{a^n b^n \mid n \geq 0\right\}
              $$
    \end{itemize}
}

\paragraph{Lenguaje libre de contexto.} Diremos que $L \subseteq \Sigma^*$ es un \textbf{lenguaje libre de contexto} si, y sólo si, existe una gramática libre de contexto $\ca{G}$ tal que:
\alignformula{
    L = \ca{L}(\ca{G})
}

\ejemplo{}{}{
    Los siguientes son lenguajes libres de contexto:
    \begin{itemize}
        \item $L=\left\{a^n b^n \mid n \geq 0\right\}$
        \item $\text{Par}=\left\{w \in\{a, b\}^* \mid w \text { tiene largo par }\right\}$
        \item $\text{Pal}=\left\{w \in\{a, b\}^* \mid w=w^{\mathrm{rev}}\right\}$
    \end{itemize}
}

\subsubsection{Árboles y derivaciones}

\paragraph{Definición.} El conjunto de \textbf{árboles ordenados y etiquetados} (o solo árboles) sobre etiquetas $\Sigma$ y $V$, se define recursivamente como:
\begin{itemize}
    \item $t:=a$ es un árbol para todo $a \in \Sigma$.
    \item si $t_1,\ldots, t_k$ son árboles, entonces $t:= X(t_1, \ldots, t_k)$ es un árbol para todo $X \in V$.
\end{itemize}

Para un árbol $t = X(t_1,\ldots,t_k)$ cualquiera se define:
\begin{itemize}
    \item $\texttt{raiz}(t)=X$
    \item $\texttt{hijos}(t)= t_1,\ldots,t_k$
\end{itemize}

Si $t = a$, entonces decimos que $t$ es una \textbf{hoja}, $\texttt{raiz}(t) = a$ y $\texttt{hijos}(t) = \epsilon$.

\paragraph{Definición.} Fije una CFG $\ca{G} = (V, \Sigma, P, S)$. Se define el conjunto de \textbf{árboles de derivación} recursivamente como:
\begin{itemize}
    \item Si $a \in \Sigma$, entonces $t = a$ es un árbol de derivación.
    \item Si $X \to X_1 \ldots X_k \in P$ y $t_1,\ldots,t_k$ son árboles de derivación con $\texttt{raiz}(t) = X_i$ para todo $i \leq k$, entonces $t = X(t_1,\ldots,t_k)$ es un árbol de derivación.
\end{itemize}

Decimos que $t$ es un \textbf{árbol de derivación de} $\ca{G}$ si:
\begin{enumerate}
    \item $t$ es un árbol de derivación y
    \item $\texttt{raiz}(t) = S$.
\end{enumerate}
Los árboles de derivación son todos los árboles que parten desde $S$.

\ejemplo{}{}{
    Sea $\ca{G}$ una CFG tal que:
    $$
        G: \quad E \; \to \; E + E \mid E * E \mid n
    $$
    Algunos árboles de derivación para $\ca{G}$ son:
    \img{img/cap4/ejemplo5.png}{0.45}
}

\paragraph{Definición.} Sea $\ca{G}$ una CFG y $w \in \Sigma^*$. Se define la función $\texttt{yield}$ sobre árboles, recursivamente como:
\begin{itemize}
    \item Si $t = a \in \Sigma$, entonces $\texttt{yield}(t) = a$.
    \item Si $t$ no es una hoja y $\texttt{hijos}(t) = t_1 t_2 \ldots t_k$, entonces:
          $$
              \texttt{yield}(t) =\texttt{yield}(t_1) \cdot \texttt{yield}(t_2) \cdot \ldots \cdot \texttt{yield}(t_k)
          $$
\end{itemize}

Decimos que $t$ es un \textbf{árbol de derivación de} $\ca{G}$ \textbf{para} $w$ si:
\begin{enumerate}
    \item $t$ es un árbol de derivación de $\ca{G}$ y
    \item $\texttt{yield}(t) = w$
\end{enumerate}
Lo anterior significa que las hojas de $t$ forman la palabra $w$.

\paragraph{Proposición.} Sea $\ca{G} = (V, \Sigma, P, S)$ una CFG y $w \in \Sigma^*$. Tenemos que:
$$
    w \in \mathcal{L}(\mathcal{G}) \quad \text { si, y solo si,} \quad \text{existe un árbol de derivación de } \mathcal{G} \text { para } w.
$$
Un árbol de derivación es la \textbf{representación gráfica} de una derivación.

\ejemplo{}{}{
    \vspace{-10pt}
    \img{img/cap4/ejemplo6.png}{0.4}
}

\paragraph{Definición.} Sea $\ca{G} = (V, \Sigma, P, S)$ una CFG.
\begin{itemize}
    \item Definimos la \textbf{derivación por la izquierda} $\underset{\mathrm{lm}}{\Rightarrow} \; \subseteq(V \cup \Sigma)^* \times(V \cup \Sigma)^*$:
          \alignformula{
              w \cdot X \cdot \beta \underset{\mathrm{lm}}{\Rightarrow} w \cdot \gamma \cdot \beta \quad \text { si, y solo si, } \quad X \rightarrow \gamma \in P
          }
          para todo $X \in V$, $w \in \Sigma^*$ y $\beta,\gamma \in (V \cup \Sigma)^*$.

    \item Definimos la \textbf{derivación por la derecha} $\underset{\mathrm{rm}}{\Rightarrow} \; \subseteq(V \cup \Sigma)^* \times(V \cup \Sigma)^*$:
          \alignformula{
              \alpha \cdot X \cdot w \underset{\mathrm{rm}}{\Rightarrow} \alpha \cdot \gamma \cdot w \quad \text { si, y solo si, } \quad X \rightarrow \gamma \in P
          }
          para todo $X \in V$, $w \in \Sigma^*$ y $\alpha,\gamma \in (V\cup \Sigma)^*$.
\end{itemize}

Se define $\overunder{\Rightarrow}{\mathrm{lm}}{*}$ y $\overunder{\Rightarrow}{\mathrm{rm}}{*}$ como la \textbf{clausura refleja y transitiva} de $\overunder{\Rightarrow}{\mathrm{lm}}{}$ y $\overunder{\Rightarrow}{\mathrm{rm}}{}$, respectivamente.  \medbreak

$\overunder{\Rightarrow}{\mathrm{lm}}{}$ y $\overunder{\Rightarrow}{\mathrm{rm}}{}$ solo reemplaza \textbf{a la izquierda} (leftmost) y \textbf{derecha} (rightmost).

\ejemplo{}{}{
    \vspace{-10pt}
    \img{img/cap4/ejemplo7.png}{0.4}
}

\paragraph{Proposicion.} Por cada árbol de derivación, existe una \textbf{única} derivación por la izquierda y una \textbf{única} derivación por la derecha. \medbreak

Por lo tanto, desde ahora podemos hablar de \textbf{árbol de derivación y derivación (izquierda o derecha)} indistintamente.

\subsubsection{Lenguajes regulares vs libres de contexto}

\paragraph{Proposición.} Para todo lenguaje regular $L$, existe una gramática libre de contexto $\ca{G}_\ca{A}$:
\alignformula{
    L = \ca{L}(\ca{G}_\ca{A})
}

\paragraph{Idea demostración.} Dado un autómata finito determinista $\ca{A} = (Q, \Sigma, \delta, q_0, F)$, ¿cómo construimos una gramática libre de contexto? \medbreak

Defina la gramática $\ca{G}_\ca{A} = (Q, \Sigma, P_\ca{A}, q_0)$ tal que:
\begin{itemize}
    \item si $\delta(p,a) = q$, entonces $p \to aq \in P\ca{A}$
    \item si $p \in F$, entonces $p \to \epsilon \in P_\ca{A}$
\end{itemize}

Queda como ejercicio propuesto al lector demostrar que $\ca{L}(\ca{A}) = \ca{L}(\ca{G}_\ca{A})$

\subsection{Simplificación de gramáticas}

¿Cómo podemos simplificar la siguiente gramática?
\begin{align*}
    G: \quad S \  & \to \ aAa \mid aBD \mid aBH \\
    A \           & \to \ B \mid D              \\
    B \           & \to \ aBa \mid b            \\
    C \           & \to \ aCC \mid bC           \\
    D \           & \to \ aDCa \mid CFa         \\
    F \           & \to \ aFDa \mid aab         \\
    H \           & \to \ \epsilon
\end{align*}
\begin{enumerate}
    \item Dada una variable $X$, ¿es $X$ \textbf{útil} para producir palabras?
    \item Dada una producción $p:X \to \gamma$, ¿es $p$ \textbf{útil} para producir palabras?
\end{enumerate}

\subsubsection{Eliminación de variables inútiles}

\paragraph{Definición.} Sea $\ca{G} = (V,\Sigma,P,S)$ una CFG. Diremos que una variable $X \in V$ es \textbf{útil} si existe una derivación:
\alignformula{
    S \stackrel{\star}{\Rightarrow} \alpha X \beta \stackrel{\star}{\Rightarrow} w
}
Al contrario, diremos que una variable $X$ es \textbf{inútil} si NO es útil.

\paragraph{Definición.} Para una variable $X \in V$:
\begin{enumerate}
    \item Decimos que $X$ es \textbf{alcanzable} si existe una derivación:
          \alignformula{
              S \stackrel{\star}{\Rightarrow} \alpha X \beta
          }
    \item Decimos que $X$ es \textbf{generadora} si existe una derivación:
          \alignformula{
              X \stackrel{\star}{\Rightarrow} w
          }
\end{enumerate}

\paragraph{Propiedad.} Para toda variable $X \in V - \{S\}$:
\alignformula{
    \text{existe una producción } Y \to \alpha X \beta \in P \text{ tal que } Y \in V \text{ es alcanzable} \quad \Leftrightarrow \quad X \text{ es alcanzable.}
}
La demostración de esta propiedad queda como ejercicio propuesto al lector. \medbreak

Un algoritmo para determinar si una variable es alcanzable:
\begin{algorithm}[hbt!]
    \setstretch{1.25}
    \DontPrintSemicolon
    \SetKwFunction{Falcanzables}{alcanzables}
    \SetKwProg{Fn}{Function}{:}{}
    \SetKwInOut{Input}{input}\SetKwInOut{Output}{output}
    \SetKw{Let}{let}
    \SetKw{Take}{take}
    \Input{Gramática $\ca{G}=(V,\Sigma,P,S)$}
    \Output{Conjunto $C$ de variables alcanzables}
    \Fn{\Falcanzables{$\ca{G}$}}{
        \Let $C_0 := \{S\}$

        \Let $C := \varnothing$

        \While{$C_0 \neq \varnothing$}{
            \Take $Y \in C_0$

            $C_0 := C_0 - \{Y\}$

            $C := C \cup \{Y\}$

            \ForEach{$X \in V - C$ \textit{tal que existe una regla} $\Big(Y \to \alpha X \beta\Big) \in P$}{
                $C_0 := C_0 \cup \{X\}$
            }
        }

        \Return $C$
    }
\end{algorithm}

\paragraph{Propiedad.} Para toda variable $X \in V$:
\alignformula{
    \text{existe una regla } X \to \alpha \text{ tal que todas las variables en } \alpha \text{ son generadoras} \quad \Leftrightarrow \quad X \text{ es generadora.}
}

La demostración de esta propiedad queda como ejercicio propuesto al lector. \medbreak

Un algoritmo para determinar si una variable es generadora:

\begin{algorithm}[hbt!]
    \setstretch{1.25}
    \DontPrintSemicolon
    \SetKwFunction{Falcanzables}{alcanzables}
    \SetKwProg{Fn}{Function}{:}{}
    \SetKwInOut{Input}{input}\SetKwInOut{Output}{output}
    \SetKw{Let}{let}
    \SetKw{Take}{take}
    \Input{Gramática $\ca{G}=(V,\Sigma,P,S)$}
    \Output{Conjunto $G$ de variables generadoras}
    \Fn{\Falcanzables{$\ca{G}$}}{
        \Let $G_0 := \{X \in V \mid (X \to w) \in P\}$

        \Let $G := \varnothing$

        \While{$G_0 \neq G$}{
            $G := G_0$

            \ForEach{$(X \to \alpha) \in P$}{
                \If{\textit{todas las variables en} $\alpha$ \textit{estan en} $G$}{
                    $G_0 := G_0 \cup \{X\}$
                }
            }
        }

        \Return $G$
    }
\end{algorithm}

\teorema{}{}{
    Sea $G = (V, \Sigma, P, S)$ una CFG. Sea $\ca{G}''$ una gramática creada a partir de $\ca{G}$ después de:
    \begin{itemize}
        \item eliminar todas las variables y reglas NO generadoras.
        \item eliminar todas las variables y reglas NO alcanzables.
    \end{itemize}
    Entonces, $\ca{L}(\ca{G}'') = \ca{L}(\ca{G})$ y $\ca{G}''$ no contiene variables inútiles.
}

\textbf{Nota:} Debemos respetar el orden para eliminar variables generadoras y luego las alcanzables. De hacerlo al revés, la gramática resultante puede no definir el mismo lenguaje que la inicial.

\paragraph{Demostración teorema 18.} Sea $\ca{G} = (V, \Sigma, P, S)$ una CFG. \medbreak

Sea $\ca{G}' = (V', \Sigma, P', S)$ al eliminar las variables \textbf{no generadoras} de $\ca{G}$:
$$
    \begin{aligned}
        V^{\prime} & =\{X \in V \mid \exists w.\ X \underset{\mathcal{G}}{\stackrel{\star}{\Rightarrow}} w\}                                 \\
        P^{\prime} & =\left\{X \rightarrow \alpha \in P \mid X \in V^{\prime} \wedge \alpha \in\left(V^{\prime} \cup \Sigma\right)^*\right\}
    \end{aligned}
$$

Sea $\ca{G}'' = (V'', \Sigma, P'', S)$ al eliminar las variables \textbf{no alcanzables} de $\ca{G}'$:
$$
    \begin{aligned}
        V^{\prime \prime} & =\left\{X \in V^{\prime} \mid \exists \alpha, \beta.\ S \underset{\mathcal{G}^{\prime}}{\stackrel{\star}{\Rightarrow}} \alpha X \beta\right\}  \\
        P^{\prime \prime} & =\left\{X \rightarrow \alpha \in P^{\prime} \mid X \in V^{\prime \prime} \wedge \alpha \in\left(V^{\prime \prime} \cup \Sigma\right)^*\right\}
    \end{aligned}
$$

Considere las siguientes propiedades de $\ca{G}$, $\ca{G}'$ y $\ca{G}''$:
\begin{enumerate}
    \item Para todo $\alpha \in (V \cup \Sigma)*$, si $\alpha \overunder{\Rightarrow}{\ca{G}}{*} w$ entonces $\alpha \overunder{\Rightarrow}{\ca{G}'}{*} w$.

    \item Para todo $\alpha \in (V' \cup \Sigma)*$, si $S \overunder{\Rightarrow}{\ca{G}'}{*} \alpha$ entonces $S \overunder{\Rightarrow}{\ca{G}''}{*} \alpha$

    \item Para todo $\alpha \in (V'' \cup \Sigma)*$, si $\alpha \overunder{\Rightarrow}{\ca{G}'}{*} w$ entonces $\alpha \overunder{\Rightarrow}{\ca{G}''}{*} w$.
\end{enumerate}

La demostración de estas propiedades queda como ejercicio propuesto al lector. \bigbreak

\textit{Demostración $\mathcal{L}\left(\mathcal{G}^{\prime \prime}\right) \subseteq \mathcal{L}(\mathcal{G})$.} Como $V'' \subseteq V$ y $P'' \subseteq P$, entonces es trivial que $\mathcal{L}\left(\mathcal{G}^{\prime \prime}\right) \subseteq \mathcal{L}(\mathcal{G})$. \bigbreak

\textit{Demostración $\mathcal{L}(\mathcal{G}) \subseteq \mathcal{L}\left(\mathcal{G}^{\prime \prime}\right)$.} Sea $w \in \ca{L}(\ca{G})$ tal que $S \overunder{\Rightarrow}{\ca{G}}{*} w$.
\begin{itemize}
    \item Por la propiedad 1, tenemos que $S \overunder{\Rightarrow}{\ca{G}'}{*} w$.
    \item Por la propiedad 2, tenemos que $S \overunder{\Rightarrow}{\ca{G}''}{*} w$.
\end{itemize}

Por lo tanto $w \in \ca{L}(\ca{G}'')$ y concluimos que $\mathcal{L}\left(\mathcal{G}^{\prime \prime}\right) \subseteq \mathcal{L}(\mathcal{G})$. \bigbreak

\textit{Demostración variables útiles.} Queremos mostrar que para todo $X \in V''$, $X$ es \textbf{útil} en $\ca{G}''$. \bigbreak

Como $X \in V''$, entonces $S \overunder{\Rightarrow}{\ca{G}'}{*} \alpha X \beta$ para algún $\alpha, \beta \in\left(V^{\prime} \cup \Sigma\right)^*$. \medbreak

Por la propiedad 2, se tiene que: $S \overunder{\Rightarrow}{\ca{G}''}{*} \alpha X \beta$ y $\alpha, \beta \in\left(V^{\prime \prime} \cup \Sigma\right)^*$. \medbreak

Como $X \in V'$ y $\alpha,\beta \in (V' \cup \Sigma)^*$, entonces existen $u, v, w$ tal que:
$$
    \alpha \overunder{\Rightarrow}{\ca{G}}{*} u, \quad X \overunder{\Rightarrow}{\ca{G}}{*} v, \quad \beta \overunder{\Rightarrow}{\ca{G}}{*} w
$$

Por la propiedad 1, se tiene que: $\alpha \overunder{\Rightarrow}{\ca{G}'}{*} u, \quad X \overunder{\Rightarrow}{\ca{G}'}{*} v, \quad \beta \overunder{\Rightarrow}{\ca{G}'}{*} w$. \medbreak

Por la propiedad 3, se tiene que: $\alpha \overunder{\Rightarrow}{\ca{G}''}{*} u, \quad X \overunder{\Rightarrow}{\ca{G}''}{*} v, \quad \beta \overunder{\Rightarrow}{\ca{G}''}{*} w$. \medbreak

Juntando todo, $S \overunder{\Rightarrow}{\ca{G}''}{*} \alpha X \beta \overunder{\Rightarrow}{\ca{G}''}{*} uvw$ y por tanto $X$ es útil en $\ca{G}''$. \hfill $\blacksquare$

\subsubsection{Eliminación de producciones inútiles}

\paragraph{Definición.} Sea $\ca{G}$ una CFG. Decimos que:
\begin{itemize}
    \item Una producción de la forma $X \to \epsilon$ es \textbf{en vacío}.
    \item Una producción de la forma $X \to Y$ es \textbf{unitaria}.
\end{itemize}

Deseamos eliminar este tipo de producciones para simplificar nuestras gramáticas, sin embargo, debemos tener cuidado con algunos detalles.

\paragraph{Proposición.} Si $\epsilon \in \ca{L}(\ca{G})$, entonces NO se pueden borrar las producciones en vacío sin alterar el lenguaje $\ca{G}$. \medbreak

Así que, desde ahora, supondremos que $\epsilon \notin \ca{L}(\ca{G})$.

\paragraph{Definición.} Sea $\ca{G} = (V, \Sigma, P, S)$ una CFG tal que $\epsilon \notin \ca{L}(\ca{G})$. Definimos a $P^*$ como el \textbf{menor conjunto de producciones} que contiene a $P$ y \textbf{cerrado bajo} las siguientes reglas:
\begin{enumerate}
    \item Si $X \to Y \in P^*$ y $Y \to \gamma \in P^*$, entonces $X \to \gamma \in P^*$.
    \item Si $X \to \epsilon \in P^*$ y $Z \to \alpha X \beta \in P^*$, entonces $Z \to \alpha \beta \in P^*$.
\end{enumerate}
Definimos $\ca{G}^* = (V, \Sigma, P^*, S)$. Entonces:
\begin{itemize}
    \item $P^*$ es finito y
    \item $\ca{L}(\ca{G}^*) = \ca{L}(\ca{G})$.
\end{itemize}

Ahora, para cualquier palabra $w \in \ca{L}(\ca{G}^*)$, sea $\ca{T}$ un árbol de derivación de $w$ en $\ca{G}^*$ de \textbf{tamaño mínimo}. Definimos las siguientes propiedades:
\begin{enumerate}
    \item El árbol de derivación $\ca{T}$ NO usa una \textbf{producción unitaria}.
    \item El árbol de derivación $\ca{T}$ NO usa una \textbf{producción en vacío}.
\end{enumerate}

La demostración de estas propiedades se hace por contradicción: se supone que $\ca{T}$ usa una producción unitaria o en vacío y se comprueba que si ocurre esto entonces $\ca{T}$ no tiene tamaño mínimo. \bigbreak

Por la propiedad 1 y 2, tenemos que \textit{para todo $w \in \ca{L}(\ca{G}^*)$, existe una derivación de $w$ en $\ca{G}$ que NO usa producciones \textbf{en vacío} ni producciones \textbf{unitarias}}. Por lo tanto, podemos eliminar las producciones en vacío y unitarias de $\ca{G}^*$.

\teorema{}{}{
    Para toda CFG $\ca{G}$ tal que $\epsilon \notin \ca{L}(\ca{G})$, sea:
    \begin{itemize}
        \item $\ca{G}^*$ la clausura de producciones unitarias y en vacío.
        \item $\hat{\ca{G}}$ el resultado de remover toda producción unitaria o en vacío de $\ca{G}^*$.
    \end{itemize}
    Entonces, $\ca{L}(\hat{\ca{G}}) = \ca{L}(\ca{G})$ y $\hat{\ca{G}}$ no tiene producciones unitarias o en vacío.
}

Resumiendo, para eliminar las producciones en vacío y unitarias de $\ca{G}$:
\begin{itemize}
    \item construimos $\ca{G}^*$ haciendo la \textbf{clausura} de producciones unitarias y en vacío,
    \item construimos $\hat{\ca{G}}$ \textbf{removiendo} todas las producciones unitarias o en vacío de $\ca{G}^*$.
\end{itemize}
Por el resultado anterior sabemos que $\ca{L}(\hat{\ca{G}}) = \ca{L}(\ca{G})$. Es importante mencionar que es posible que $\hat{\ca{G}}$ contenga símbolos inútiles.

\subsection{Forma normal de Chomsky}

\paragraph{Definición.} Una gramática $\ca{G}$ esta en \textbf{forma normal de Chomsky} (CNF) si todas sus reglas son de la forma:
\begin{itemize}
    \item $X \to YZ$
    \item $X \to a$
\end{itemize}
\ejemplo{}{}{
    La siguiente gramática está en CNF:
    \begin{align*}
        S \  & \to \ AB \mid AC \mid SS \\
        C \  & \to \ SB                 \\
        A \  & \to a                    \\
        B \  & \to b
    \end{align*}
}
Toda gramática se puede convertir en CNF. Para facilitar el proceso, considere $\ca{G} = (V, \Sigma, P, S)$ una CFG tal que $\epsilon \notin \ca{L}(\ca{G})$.
\begin{itemize}
    \item Primero, suponga que $\ca{G}$ no contiene reglas en vacío o unitarias.
    \item Por lo tanto, todas las reglas en $\ca{G}$ son de la forma:
          \begin{itemize}
              \item $X \to \gamma \quad $ para $|\gamma| \geq 2$
              \item $X \to a$
          \end{itemize}
\end{itemize}

Los pasos para convertir $\ca{G}$ en CNF son:
\begin{enumerate}
    \item Convertir todas las reglas a la forma:
          \begin{itemize}
              \item $X \to Y_1 Y_2 \ldots Y_k$ para $k \geq 2$
              \item $X \to a$
          \end{itemize}

    \item Convertir todas las reglas a la forma:
          \begin{itemize}
              \item $X \to YZ$
              \item $X \to a$
          \end{itemize}
\end{enumerate}

Para realizar el paso 1:
\begin{itemize}
    \item Para cada $a \in \Sigma$, agregamos una nueva variable $X_a$ y una regla $X_a \to a$.
    \item Reemplazamos todas las ocurrencias antiguas de $a$ por $X_a$.
\end{itemize}

\ejemplo{}{}{
    Considere la gramática $S \to aSb \mid ab$, entonces, hacer el paso 1 termina en:
    \begin{align*}
        S \  & \to \ ASB \mid AB \\
        A \  & \to \ a           \\
        B \  & \to \ b
    \end{align*}
}

\paragraph{Correctitud.} Si $\ca{G}'$ es la gramática resultante del paso 1, entonces se cumple que $\ca{L}(\ca{G}') = \ca{L}(\ca{G})$. \bigbreak

Para realizar el paso 2, tomamos cada regla $p: X \to Y_1 Y_2 \ldots Y_k$ con $k \geq 3$ y:
\begin{itemize}
    \item Agregamos una \textbf{nueva} variable $Z$.
    \item Reemplazamos la regla $p$ por \textbf{dos reglas}:
          $$
              X \to Y_1 Z \quad \text{y} \quad Z \to Y_2 \ldots Y_k
          $$
\end{itemize}
Repetimos este paso hasta llegar a la forma normal de Chosmky.

\ejemplo{}{}{
    El resultado del paso 1 anterior es:
    \begin{align*}
        S \  & \to \ ASB \mid AB \\
        A \  & \to \ a           \\
        B \  & \to \ b
    \end{align*}
    Al realizar el paso 2, la gramática queda de la forma:
    \begin{align*}
        S \  & \to \ AZ \mid AB \\
        Z \  & \to SB           \\
        A \  & \to \ a          \\
        B \  & \to \ b
    \end{align*}
}

\paragraph{Correctitud.} Si $\ca{G}''$ es la gramática resultante del paso 2, entonces se cumple que $\ca{L}(\ca{G}'') = \ca{L}(\ca{G}')$.

\teorema{}{}{
    Sea $\ca{G} = (V, \Sigma, P, S)$ una CFG tal que $\epsilon \notin \ca{L}(\ca{G})$. Existe una gramática $\ca{G}'$ en forma normal de Chomsky tal que:
    $$
        \ca{L}(\ca{G}') = \ca{L}(\ca{G})
    $$
}

Si $\ca{G}'$ no tiene reglas unitarias ni en vacío, entonces $\ca{G}'$ es de \textbf{tamaño polinomial} con respecto a $\ca{G}$.

\subsection{Lema de bombeo para lenguajes libres de contexto}

Similarmente para los lenguajes regulares, existe un lema de bombeo para lenguajes libres de contexto. \bigbreak

Sea $L\subseteq \Sigma^*$. Si $L$ es \textbf{libre de contexto}, entonces:
\alignformula{
    (\text{LB}^\text{CFL}) \quad &\text{existe un } N > 0 \text{ tal que}\\
    &\text{para toda palabra } z \in L \text{ con } |z| \geq N \\
    &\text{existe una descomposición } z = uvwxy \\
    &\hphantom{aaaaa} \text{con } vx \neq \epsilon \text{ y } |vwx| \leq N \text{ tal que}\\
    &\text{para todo } i \geq 0,\; u \cdot v^i \cdot w \cdot x^i \cdot y \in L
}

Análogamente, para demostrar que un lenguaje $L$ NO es libre de contexto, usamos el contrapositivo del lema. \bigbreak

Sea $L \subseteq \Sigma^*$. Si:
\alignformula{
    (\neg\text{LB}^\text{CFL}) \quad &\text{para todo } N > 0 \text{ tal que}\\
    &\text{existe una palabra } z \in L \text{ con } |z| \geq N \\
    &\text{para toda descomposición } z = uvwxy \\
    &\hphantom{aaaaa} \text{con } vx \neq \epsilon \text{ y } |vwx| \leq N \text{ tal que}\\
    &\text{existe } i \geq 0,\; u \cdot v^i \cdot w \cdot x^i \cdot y \notin L\\
    &\hphantom{aaaaa} \text{entonces } L \text{ NO es libre de contexto}.
}
\paragraph{Jugando contra un demonio.} El lema de bombeo puede verse como el siguiente ``juego'':
\img{img/cap4/demonio.png}{0.3}

\ejemplo{}{}{
    Considere el lenguaje $L = \{ a^{n^2} \mid n > 0\}$
    \img{img/cap4/ejemplo11.png}{0.25}

    Ganamos ya que al bombear con $i = 2$ ya no se cumple que $j + k + l + m + n = N^2$ (se rompe el equilibrio), y entonces $z \notin L$.
}

\newpage

\ejemplo{}{}{
    Considere el lenguaje $L = \{a^n b^n c^n \mid n > 0\}$.
    \img{img/cap4/ejemplo12.png}{0.25}

    Ganamos el juego, ya que, como $uvwxy = a^N b^N c^N$ con $vx \neq \epsilon$ y $|vwx| \leq N$, entonces
    $$
        vwx \in \ca{L}(a^* b^*) \quad \text{o} \quad vwx \in \ca{L}(b^* c^*)
    $$
    \begin{itemize}
        \item Si $vwx \in \ca{L}(a^+ b^+)$, entonces:
              \begin{itemize}
                  \item $|u v^2 w x^2 y|_{a,b} > 2N$
                  \item $|u v^2 w x^2 y|_{c} = N$
              \end{itemize}
              Por lo tanto, $z' \notin L$.

        \item Si $vwx \in \ca{L}(b^+ c^+)$, entonces:
              \begin{itemize}
                  \item $|u v^2 w x^2 y|_{b,c} > 2N$
                  \item $|u v^2 w x^2 y|_{a} = N$
              \end{itemize}
              Por lo tanto, $z' \notin L$.
    \end{itemize}
    En ambos casos, $uv^2 w x^2 y \notin L$, y por tanto $L$ no es CFG.
}

\paragraph{Lema versión juego.} \textit{``Dado un lenguaje $L \subseteq \Sigma^*$, si \textbf{UNO} tiene una estrategia ganadora en el juego $(\neg\text{LB}^\text{CFL})$ para toda estrategia posible del demonio, entonces $L$ \textbf{NO} es libre de contexto''.}

\paragraph{Consecuencias.} Podemos establecer la siguiente proposición en base al lema de bombeo:
\begin{itemize}
    \item Para todos lenguajes libres de contexto $L_1$ y $L_2$, se cumple que $L_1 \cup L_2$ es un lenguaje libre de contexto.
    \item Existen lenguajes libres de contexto $L$, $L_1$ y $L_2$ tales que:
          \begin{itemize}
              \item $L_1 \cap L_2$ \textbf{NO} es un lenguaje libre de contexto.
              \item $L^C$ \textbf{NO} es un lenguaje libre de contexto.
          \end{itemize}
\end{itemize}

Para demostrar que la intersección no es libre de contexto, basta con un contraejemplo. Tomemos $L_1 = \{a^n b^n c^m \mid n\geq 0, m \geq 0\}$ y $L_2 = \{a^m b^n c^n \mid n \geq 0, m \geq 0\}$, al hacer su intersección, obtendremos el lenguaje $L = \{a^n b^n c^n \mid n \geq 0\}$, que ya vimos que no es libre de contexto. \medbreak

Queda como ejercicio propuesto para el lector demostrar que $L^C$ no es libre de contexto.

\subsection{Algoritmo CKY}

Dado un lenguaje libre de contexto $L$ y una palabra $w$, ¿cómo verificamos si $w \in L$?
\begin{itemize}
    \item Convertimos $\ca{G}$ en forma normal de Chomsky.
    \item Probamos todas las derivaciones de altura a lo más $|w| + 1$.
    \item Si encontramos una derivación retornamos $\texttt{TRUE}$.
\end{itemize}

Los pasos anteriores se conoce como el algoritmo CKY, inventado por John Cocke, Tadao Kasami y Daniel Younger. Es un algoritmo que usa \textbf{programación dinámica}, y es \textbf{cúbico} en $|w|$ y \textbf{lineal} en $|\ca{G}|$:
\alignformula{
    \text{Tiempo:}\; \ca{O}(|w|^3 \cdot |\ca{G}|)
}
Por simplicidad, asumiremos que las gramáticas que recibe el algoritmo están en \textbf{Forma Normal de Chomsky} (CNF).

\paragraph{Tabla del algoritmo CKY.} Para una palabra $w = a_1 a_2 \ldots a_n$ y una gramática $\ca{G} = (V, \Sigma, P, S)$ construimos la \textbf{tabla CKY}:

\img{img/cap4/cky1.png}{0.35}

Para todo $1 \leq 1 \leq j \leq n$ se define:
$$
    C_{i\, j} = \{X \in V \mid X \overunder{\Rightarrow}{\ca{G}}{*} a_i \ldots a_j\}
$$

\begin{multicols}{2}

    \paragraph{Paso 0 (inicial).} Para cada $i$, construimos el conjunto $C_{ii} \subseteq V$ tal que:
    $$
        C_{i\,i} = \{X \in V \mid X \to a_i \in P\}
    $$
    \img{img/cap4/cky2.png}{0.3}

    \paragraph{Paso 1.} Para cada $i$, construimos el conjunto $C_{i\, i +1} \subseteq V$ tal que:
    \begin{align*}
        C_{i\, i+1} =\; & \{X \in V \mid X \to YZ \in P \text{ para algún } \\
                        & Y \in C_{ii} \wedge Z \in C_{i+1 i+1}\}
    \end{align*}

    \img{img/cap4/cky3.png}{0.3}
\end{multicols}

\paragraph{Paso $k$ $(k > 0)$.} Para cada $i$, construimos el conjunto $C_{i\, i+k} \subseteq V$ tal que:
$$
    C_{i\, i+k} = \{ X \in V \mid \exists j \in [i,i+k].\ X \to YZ \in P \text{ para algún } Y \in C_{i\, j} \wedge Z \in C_{j+i \, i+k}\}
$$
\begin{figure}[H]
    \centering
    \includegraphics[scale=0.35]{img/cap4/cky4.png}
    \includegraphics[scale=0.35]{img/cap4/cky5.png}
\end{figure}

\ejemplo{}{}{
    Considere la palabra $bbaba$ y la gramática:
    \img{img/cap4/ejemplo13.png}{0.6}

    Considere la palabra $abba$ y la gramática:
    \img{img/cap4/ejemplo13_1.png}{0.6}
}
\newpage
\paragraph{Algoritmo CKY.} A continuación se muestra el pseudo-código del algoritmo:

\begin{algorithm}[hbt!]
    % \caption*{An algorithm with caption}\label{alg:two}
    \setstretch{1.5}
    \DontPrintSemicolon
    \SetKwFunction{FAlgoritmoCKY}{AlgoritmoCKY}
    \SetKwProg{Fn}{Function}{:}{}
    \SetKwInOut{Input}{input}\SetKwInOut{Output}{output}
    \SetKw{Let}{let}
    \SetKw{Check}{check}
    \Input{Una gramática $\ca{G}=(V,\Sigma,P,S)$ y una palabra $w = a_1 a_2 \ldots a_n$}
    \Output{$\texttt{TRUE}$ si, y sólo si, $w \in \ca{L}(\ca{G})$}
    \Fn{\FAlgoritmoCKY{$\ca{G},w$}}{
        \For{$i = 1$ \KwTo $n$}{
            \Let $C_{i\, i} = \varnothing$

            \For{$X \to C \in P$}{
                \If{$c = a_i$}{\Let $C_{i\, i} = C_{i\, i} \cup \{X\}$}
            }
        }

        \For{$k=1$ \KwTo $n - 1$}{
            \For{$i=1$ \KwTo $n-k$}{
                \Let $C_{i\, i+k} = \varnothing$

                \For{$j = i$ \KwTo $i+k-1$}{
                    \For{$X \to YZ \in P$}{
                        \If{$Y \in C_{i\, j} \wedge Z \in C_{j+1\, i+k}$}{
                            \Let $C_{i\, i+k} = C_{i\, i+k} \cup \{X\}$
                        }
                    }
                }
            }
        }

        \Return \Check $S \in C_{1\, n}$
    }
\end{algorithm}

\paragraph{Análisis algoritmo CKY.} En su correctitud, para toda gramática $\ca{G}$ y para toda palabra $w \in \Sigma^*$ se tiene que:
$$
\texttt{AlgoritmoCKY}(\ca{G},w) = \texttt{TRUE} \quad \Leftrightarrow \quad w \in \ca{L}(\ca{G})
$$

La demostración queda como ejercicio propuesto al lector. \bigbreak

Si el input es de tamaño $|w|$ y la gramática es de tamaño $|\ca{G}|$, entonces:
$$
\text{Tiempo del algoritmo CKY:} \quad \ca{O}(|w|^3 \cdot |\ca{G}|)
$$
% \section{Algoritmos para lenguajes libres de contexto}
\subsection{Autómatas apiladores}
\subsubsection{Versión normal}

\fig{img/cap5/idea_automata.png}{0.7}{Idea de un autómata apilador}

\paragraph*{Definición.} Un autómata apilador (\textit{PushDown Automata}, PDA) es una estructura:
\alignformula{
    \ca{P}=(Q,\Sigma,\Gamma,\Delta,q_0,\bot,F)
}
\begin{itemize}
    \item $Q$ es un conjunto finito de \textbf{estados}.
    \item $\Sigma$ es el alfabeto del \textbf{input}.
    \item $q_0 \in Q$ es el estado \textbf{inicial}.
    \item $F$ es el conjunto de estados \textbf{finales}.
    \item $\Gamma$ es el alfabeto de \textbf{stack}.
    \item $\bot \in \Gamma$ es el símbolo \textbf{inicial del stack} (fondo).
    \item $\Delta \subseteq(Q \times(\Sigma \cup\{\epsilon\}) \times \Gamma) \times\left(Q \times \Gamma^*\right)$ es una relación finita de transición.
\end{itemize}

Intuitivamente, la transición:
\alignformula{
    \Big((p,a,A),(q,B_1B_2\cdots B_k)\Big) \in \Delta
}
si el autómata apilador está:
\begin{itemize}
    \item en el estado $p$, leyendo $a$, y en el tope del stack hay una $A$,
\end{itemize}
entonces:
\begin{itemize}
    \item cambia al estado $q$, y modifico el tope $A$ por $B_1B_2\cdots B_k$.
\end{itemize}

Intuitivamente, la transición \textbf{en vacío}:
\alignformula{
    \Big((p,\epsilon,A),(q,B_1B_2\cdots B_k)\Big) \in \Delta
}
si el autómata apilador está:
\begin{itemize}
    \item en el estado $p$, \textit{sin lectura de una letra}, y en el tope del stack hay una $A$,
\end{itemize}
entonces:
\begin{itemize}
    \item cambia al estado $q$, y modifico el tope $A$ por $B_1B_2\cdots B_k$.
\end{itemize}

\ejemplo{}{}{
    $$
        \ca{P}=(Q,\Sigma,\Gamma,\Delta,q_0,\bot,\{q_f\})
    $$
    \begin{itemize}
        \item $Q=\{q_0,q_1,q_f\}$, $\Sigma = \{a,b\}$, $\Gamma = \{A,\bot\}$ y $\Delta$:
              $$
                  \begin{array}{ll}
                      \left(q_0, a, \perp, q_0, A \perp\right)         & q_0 \perp \stackrel{a}{\rightarrow} q_0 A \perp \\
                      \left(q_0, a, A, q_0, A A\right)                 & q_0 A \stackrel{a}{\rightarrow} q_0 A A         \\
                      \left(q_0, b, A, q_1, \epsilon\right)            & q_0 A \stackrel{b}{\rightarrow} q_1             \\
                      \left(q_1, b, A, q_1, \epsilon\right)            & q_1 A \stackrel{b}{\rightarrow} q_1             \\
                      \left(q_1, \epsilon, \perp, q_f, \epsilon\right) & q_1 \perp \stackrel{\epsilon}{\rightarrow} q_f
                  \end{array}
              $$
    \end{itemize}

    \img{img/cap5/ejemplo1.png}{0.65}
}

\paragraph*{Notación.} Dada una palabra $A_1A_2\ldots A_k \in \Gamma^+$ decimos que:
\begin{itemize}
    \item $A_1 A_2 \ldots A_k$ es un stack (contenido),
    \item $A_1$ es el \textbf{tope} del stack y
    \item $A_2 \ldots A_k$ es la \textbf{cola} del stack.
\end{itemize}

\paragraph*{Definición.} Una \textbf{configuración} de $\ca{P}$ es una tupla $(q\cdot \gamma, w) \in (Q\cdot \Gamma^*, \Sigma^*)$ tal que:
\begin{itemize}
    \item $q$ es el estado actual.
    \item $\gamma$ es el contenido del stack.
    \item $w$ es el contenido del input.
\end{itemize}
Decimos que una configuración:
\alignformula{
    (q\cdot \gamma, w) \in (Q\cdot \Gamma^*, \Sigma^*)
}
\begin{itemize}
    \item es \textbf{inicial} si $q\cdot \gamma = q_0\cdot \bot$.
    \item es \textbf{final} si $q\cdot \gamma = q_f\cdot \epsilon$ con $q_f \in F$ y $w=\epsilon$.
\end{itemize}

\paragraph*{Definición.} Se define la relación $\vdash_{\ca{P}}$ de \textbf{siguiente-paso} entre configuraciones de $\ca{P}$:
\alignformula{
    \left(q_1 \cdot \gamma_1, w_1\right) \quad \vdash_{\mathcal{P}} \quad\left(q_2 \cdot \gamma_2, w_2\right)
}
si, y sólo si, existe una transición $\left(q_1, a, A, q_2, \alpha\right) \in \Delta \text { y } \gamma \in \Gamma^*$ tal que:
\begin{itemize}
    \item $w_1 = a \cdot w_2$
    \item $\gamma_1 = A\cdot \gamma$
    \item $\gamma_2 = \alpha \cdot \gamma$
\end{itemize}

Se define $\vdash_{\ca{P}}^*$ como la clausura \textbf{refleja} y \textbf{transitiva} de $\vdash_\ca{P}$. En otras palabras:
\alignformula{
    \begin{gathered}
        \left(q_1 \gamma_1, w_1\right) \vdash_{\mathcal{P}}^*\left(q_2 \gamma_2, w_2\right) \text { si uno puede ir de }\left(q_1 \gamma_1, w_1\right) \text { a }\left(q_2 \gamma_2, w_2\right) \\
        \text { en } 0 \text { o más pasos. }
    \end{gathered}
}
\ejemplo{}{}{
    Para la palabra $w=aaabbb$, tenemos la ejecución:
    \img{img/cap5/ejemplo2.png}{0.7}
}

\paragraph*{Definiciones.} $\cal{P}$ \textbf{acepta} $w$ si, y sólo si, $\left(q_0 \perp, w\right) \vdash_{\mathcal{P}}^*\left(q_f, \epsilon\right)$ para algún $q_f \in F$.

\hspace{70pt} El \textbf{lenguaje aceptado} por $\ca{P}$ se define como:
\alignformula{
    \ca{L}(\ca{P})=\{w\in \Sigma^*\| \ \ca{P} \text{ acepta } w\}
}
\ejemplo{}{}{
    El lenguaje aceptado por el PDA utilizado en los ejemplos anteriores es $\ca{L}(\ca{P})=\{a^nb^n\ | \ n \ge 0\}$.
}

\subsubsection{Versión alternativa}
Esta definición de autómata apilador es poco común pero trae algunas ventajas:
\begin{itemize}
    \item Es un modelo que ayuda a entender mejor los algoritmos de evaluación para gramáticas.
    \item Es un modelo menos estándar pero mucho más sencillo.
    \item Al profe Cristian le gustó y lo encontró interesante.
\end{itemize}

\paragraph*{Definición.} Un \textbf{PDA alternativo} es una estructura:
\alignformula{
    \ca{D}=(Q,\Sigma,\Delta,q_0,F)
}
\begin{itemize}
    \item $Q$ es un conjunto finito de \textbf{estados}.
    \item $\Sigma$ es el alfabeto del \textbf{input}.
    \item $q_0 \in Q$ es el estado \textbf{inicial}.
    \item $F$ es el conjunto de estados \textbf{finales}.
    \item $\Delta \subseteq Q^+ \times (\Sigma \cup \{\epsilon\})\times Q^*$ es una \textbf{relación finita de transición}.
\end{itemize}
Intuitivamente, la transición:
\alignformula{
    \Big( A_1\ldots A_i, a, B_1 \ldots B_j \Big) \in \Delta
}
si el autómata apilador tiene:
\begin{itemize}
    \item $A_1\ldots A_i$ en el tope del stack y leyendo $a$,
\end{itemize}
entonces:
\begin{itemize}
    \item cambia el tope $A_1\ldots A_i$ por $B_1\ldots B_j$.
\end{itemize}

En este tipo de autómata apilador, \textbf{no hay diferencia} entre estados y alfabeto del stack.

\paragraph*{Definición.} Una \textbf{configuración} de $\ca{D}$ es una tupla
\alignformula{
    (q_1\ldots q_k, w) \in (Q^+,\Sigma^*)
}
tal que:
\begin{itemize}
    \item $q_1\ldots q_k$ es el contenido del stack con $q_1$ el tope del stack.
    \item $w$ es el contenidod el input.
\end{itemize}
Decimos que una configuración:
\begin{itemize}
    \item $(q_0,w)$ es \textbf{inicial}.
    \item $(Q_f,\epsilon)$ es \textbf{final} si $q_f \in F$.
\end{itemize}

\paragraph*{Definición.} Se define la relación $\vdash_{\ca{D}}$ de \textbf{siguiente-paso} entre configuraciones de $\ca{D}$:
\alignformula{
    \left( \gamma_1, w_1\right) \quad \vdash_{\mathcal{D}} \quad\left(\gamma_2, w_2\right)
}
si, y sólo si, existe una transición $\left(\alpha, a, \beta\right) \in \Delta \text { y } \gamma \in \Gamma^*$ tal que:
\begin{itemize}
    \item $w_1 = a \cdot w_2$
    \item $\gamma_1 = \alpha\cdot \gamma$
    \item $\gamma_2 = \beta \cdot \gamma$
\end{itemize}

Se define $\vdash_{\ca{D}}^*$ como la clausura \textbf{refleja} y \textbf{transitiva} de $\vdash_\ca{D}$.

\paragraph*{Definiciones.} $\ca{D}$ \textbf{acepta} $w$ si, y sólo si, $(q_0,w) \vdash_\ca{D}^* (q_f,\epsilon)$ para algún $q_f \in F$. Además, el \textbf{lenguaje aceptado} por $\ca{D}$ se define como:
\alignformula{
    \ca{L}(\ca{D})=\{w\in \Sigma^*\| \ \ca{D} \text{ acepta } w\}
}

\newpage
\ejemplo{}{}{
    $$
        \ca{D}=(Q,\{a,b\},\Delta,q_0,F)
    $$
    \begin{itemize}
        \item $Q=\{\bot, q_0, q_1, q_f\}$ y $\Delta$:
              \img{img/cap5/ejemplo4.png}{0.6}
    \end{itemize}
    $$
        \ca{L}(\ca{D})=\{a^nb^n \ |\ n\ge 1\}
    $$
}

\teorema{}{}{
    Para todo autómata apilador $\ca{P}$ existe un autómata apilador alternativo $\ca{D}$, y viceversa, tal que:
    $$
        \ca{L}(\ca{P}) = \ca{L}(\ca{D})
    $$
}
El teorema anterior nos dice que podemos usar ambos modelos de manera \textbf{equivalente}.

\subsection{Autómatas apiladores vs gramáticas libres de contexto}
¿En qué se parecen CFG a PDA?
\fig{img/cap5/cfg_vs_pda.png}{0.3}{Gramáticas vs Autómatas apiladores}

\teorema{}{}{
    Todo \textbf{lenguaje libre de contexto} puede ser descrito equivalentemente por:
    \begin{itemize}
        \item Una gramática libre de contexto (\textbf{CFG}).
        \item Un autómata apilador (\textbf{PDA}).
    \end{itemize}
}

\subsubsection{Desde CFG a PDA}
Partimos enunciado un teorema:
\teorema{}{}{
    Para toda gramática libre de contexto $\ca{G}$, existe un \textbf{autómata apilador alternativo} $\ca{D}$, tal que:
    $$
        \ca{L}(\ca{G}) = \ca{L}(\ca{D})
    $$
}

\paragraph*{Construcción $\ca{D}$ desde $\ca{G}$.} Sea $\ca{G}=(V,\Sigma,P,S)$ una CFG. Construimos un PDA alternativo $\ca{D}$ que acepta $\ca{L}(\ca{G})$:
\alignformula{
    \ca{D}=\Big( V \cup \Sigma \cup \{q_0,q_f\}, \Sigma, \Delta, q_0, \{q_f\} \Big)
}
La relación de transición $\Delta$ se define como:
\begin{table}[H]
    \centering
    \begin{tabular}{lllll}
        $\Delta$ & $=$ & $\{ (q_0, \epsilon, S \cdot q_f) \}$               & $\cup$ &                     \\
                 &     & $\{ (X,\epsilon,\gamma)\ | \ X\to \gamma \in P \}$ & $\cup$ & \textbf{(Expandir)} \\
                 &     & $\{ (a,a,\epsilon) \ | \ a \in \Sigma \}$          &        & \textbf{(Reducir)}  \\
                 &     &                                                    &        &
    \end{tabular}
\end{table}
\paragraph*{Demostración $\ca{L}(\ca{G}) = \ca{L}(\ca{D})$.} Debemos demostrar dos direcciones: $\ca{L}(\ca{G}) \subseteq \ca{L}(\ca{D})$ y $\ca{L}(\ca{D}) \subseteq \ca{L}(\ca{G})$.

\paragraph*{Demostración $\ca{L}(\ca{G}) \subseteq \ca{L}(\ca{D})$.} Para cada $w \in \ca{L}(\ca{G})$ debemos encontrar una ejecución de aceptación de $\ca{D}$ sobre $w$. ¿Cómo encontramos esta ejecución? La idea es que para cada árbol de derivación $\ca{T}$ de $\ca{G}$ sobre $w$, construimos una ejecución de $\ca{D}$ sobre $w$ que recorre el árbol $\ca{T}$ \textbf{en profundidad} (DFS). Por tanto, debemos usar \textbf{inducción} sobre la altura del árbol $\ca{T}$.

\paragraph*{Hipótesis de inducción.} Para todo árbol de derivación $\ca{T}$ de $\ca{G}$ con \textbf{altura} $h$ tal que:
\begin{itemize}
    \item la raíz de $\ca{T}$ es $X$, y
    \item $\ca{T}$ produce la palabra $w$
\end{itemize}
entonces $(X\cdot\gamma, w) \vdash_\ca{D}^* (\gamma, \epsilon)$ para todo $\gamma \in Q^+$.

\paragraph{Caso base: $h=1$.} Si $\ca{T}$ tiene altura $1$, entonces:
\begin{itemize}
    \item $\ca{T}$ produce la palabra $w=a$ para algún $a\in \Sigma$ y
    \item $\ca{T}$ consiste de un nodo $X$ y un hijo $a$ con $X \to a$.
\end{itemize}
Entonces para todo $\gamma \in Q^+$:
$$
    (X \cdot \gamma, a) \vdash_\mathcal{D} (a \cdot \gamma, a) \vdash_\mathcal{D}(\gamma, \epsilon)
$$
es una ejecución de $\ca{D}$ sobre $a$.

\paragraph*{Caso inductivo: $h=n$.} Suponemos que el árbol de derivación $\ca{T}$ de $\ca{G}$ tiene \textbf{altura} $n$ tal que:
\begin{itemize}
    \item la raíz de $\ca{T}$ es $X$, y
    \item $\ca{T}$ produce la palabra $w$.
\end{itemize}
\textbf{Sin pérdida de generalidad}, suponga que $\ca{T}$ es de la forma:
\img{img/cap5/dem1.png}{0.5}
donde $w = u\cdot v$ y $X\to YZ$. Por HI, se tiene que para todo $\gamma_1, \gamma_2 \in Q^+$:
$$
    \begin{aligned}
        \left(Y \cdot \gamma_1, u\right) & \vdash_{\mathcal{D}}^*\left(\gamma_1, \epsilon\right) \\
        \left(Z \cdot \gamma_2, v\right) & \vdash_{\mathcal{D}}^*\left(\gamma_2, \epsilon\right)
    \end{aligned}
$$
Para $\gamma \in Q^+$ \textbf{construimos} la siguiente ejecución de $\ca{D}$ sobre $w=uv$:
$$
    (X \cdot \gamma, u v) \vdash_{\mathcal{D}}(Y Z \cdot \gamma, u v) \vdash_{\mathcal{D}}^*(Z \cdot \gamma, v) \vdash_{\mathcal{D}}^*(\gamma, \epsilon)
$$
\hfill $\blacksquare$

La demostración de $\ca{L}(\ca{D}) \subseteq \ca{L}(\ca{G})$ se deja como ejercicio propuesto al lector.

% \paragraph*{Demostración $\ca{L}(\ca{D}) \subseteq \ca{L}(\ca{G})$.} Para cada $w \in \ca{L}(\ca{D})$ debemos encontrar un árbol de derivación de $\ca{G}$ para $w$. ¿Cómo encontramos un árbol de derivación para $w$? La idea es que si tenemos una ejecución de $\ca{D}$ sobre $w$ de la forma:
% $$
%     \left(X \cdot q_f, w\right) \vdash_{\mathcal{D}}^*\left(q_f, \epsilon\right)
% $$
% entonces $X \underset{\mathcal{G}}{\stackrel{\star}{\Rightarrow}} w$. Por tanto, podemos usar \textbf{inducción} en la cantidad de pasos de la ejecución.

% \paragraph{Hipótesis de inducción.} Para toda ejecución de $\ca{D}$ sobre $w$ de largo $k$ de la forma:
% $$
%     \left(X \cdot q_f, w\right)=\left(\gamma_0, w_0\right) \vdash_\mathcal{D}\left(\gamma_1, w_1\right) \vdash_\mathcal{D} \cdots \vdash_\mathcal{D}\left(\gamma_k, w_k\right)=\left(q_f, \epsilon\right)
% $$
% entonces $X \underset{\mathcal{G}}{\stackrel{\star}{\Rightarrow}} w$.

\subsubsection{Desde PDA a CFG}
Partimos enunciando el siguiente teorema:
\teorema{}{}{
    Para todo autómata apilador $\ca{P}$, existe una gramática libre de contexto $\ca{G}$ tal que:
    $$
        \ca{L}(\ca{P}) = \ca{L}(\ca{G})
    $$
}
\paragraph{Demostración $\ca{L}(\ca{P}) = \ca{L}(\ca{G})$.} Sea $\ca{P}=(Q,\Sigma,\Gamma,\Delta,q_0,\bot,F)$ un PDA (normal). Los pasos a seguir son:
\begin{enumerate}
    \item Convertir $\ca{P}$ a un PDA $\ca{P}'$ con \textbf{UN solo estado}.
    \item Convertir $\ca{P}'$ a una gramática libre de contexto $\ca{G}$.
\end{enumerate}

\paragraph{Paso 1.} Sea $\ca{P}=(Q,\Sigma,\Gamma,\Delta,q_0,\bot,F)$ un PDA. Podemos analizar:
\begin{itemize}
    \item ¿Por qué NO necesitamos la información de los estados?
    \item ¿Cómo guardamos la información de los estados en el stack?
\end{itemize}

Esto conlleva a la siguiente pregunta: \textit{Si el PDA está en el estado $p$ y en el tope del stack hay una $A$, ¿a cuál estado llegaré al remover $A$ del stack?} \medbreak

La solución a esta pregunta es que podemos \textbf{adivinar} (no-determinismo) el estado que vamos a llegar cuando removamos $A$ del stack. \medbreak

\textbf{Sin pérdida de generalidad}, podemos asumir que
\begin{enumerate}
    \item Todas las transiciones son de la forma:
          $$
              q A \stackrel{c}{\rightarrow} p B_1 B_2 \quad \text{o} \quad q A \stackrel{c}{\rightarrow} p \epsilon
          $$
          con $c \in (\Sigma \cup \{e\})$.
    \item Existe $q_f \in Q$ tal que si $w \in \ca{L}(\ca{P})$ entonces:
          $$
              \left(q_0 \bot, w\right) \vdash_{\mathcal{D}}^*\left(q_f, \epsilon\right)
          $$
\end{enumerate}
Estos dos puntos nos aseguran  que siempre llegamos al \textbf{mismo estado} $q_f$. Luego, construimos el autómata apilador $\ca{P}'$ con \textbf{un solo estado}:
$$
    \mathcal{P}^{\prime}=\left(\{q\}, \Sigma, \Gamma^{\prime}, \Delta^{\prime},\{q\}, \perp^{\prime},\{q\}\right)
$$
\begin{itemize}
    \item $\Gamma'= Q\times \gamma \times Q$.

          \textit{``$(p, A, q) \in \Gamma'$ si desde $p$ leyendo $A$ en el tope del stack llegamos a $q$ al hacer pop de $A$''.}

    \item $\bot' = (q_0,\bot, q_f)$.

          \textit{``El autómata parte en $q_0$ y al hacer pop de $\bot$ llegará a $q_f$''.}

    \item Si $p A \stackrel{c}{\rightarrow} p^{\prime} B_1 B_2 \in \Delta$ con $c \in (\Sigma \cup \{\epsilon\})$, entonces \textbf{para todo} $p_1,p_2 \in Q$:
          $$
              q\left(p, A, p_2\right) \stackrel{c}{\rightarrow} q\left(p^{\prime}, B_1, p_1\right)\left(p_1, B_2, p_2\right) \in \Delta^{\prime}
          $$

    \item Si $p A \stackrel{c}{\rightarrow} p^{\prime}\in \Delta$ con $c \in (\Sigma \cup \{\epsilon\})$, entonces:
          $$
              q\left(p, A, p^{\prime}\right) \stackrel{c}{\rightarrow} q \in \Delta^{\prime}
          $$
\end{itemize}

\paragraph{Hipótesis de inducción (en el número de pasos $n$).} Para todo $p,p' \in Q$, $A \in \Gamma$ y $w \in \Sigma^*$ se cumple que:
$$
    (p A, w) \vdash_{\mathcal{P}}^n\left(p^{\prime}, \epsilon\right) \quad \text { si, y solo si, } \quad\left(q\left(p, A, p^{\prime}\right), w\right) \vdash_{\mathcal{P}^{\prime}}^n(q, \epsilon)
$$
donde $\vdash_\ca{P}^n$ es la relación de \textbf{siguiente-paso} de $\ca{P}$ $n$-veces. \medbreak

Si demostramos esta hipótesis, habremos demostrado que $\ca{L}(\ca{P}) = \ca{L}(\ca{P'})$. ¿Por qué?

\paragraph{Caso base: $n=1$.} Para todo $p,p' \in Q$, y $A \in \Gamma$ se cumple que:
$$
    (p A, c) \vdash_\mathcal{P}\left(p^{\prime}, \epsilon\right) \quad \text { si, y solo si, } \quad\left(q\left(p, A, p^{\prime}\right), c\right) \vdash_{\mathcal{P}^{\prime}}(q, \epsilon)
$$
para todo $c \in (\Sigma \cup \{\epsilon\})$.

\paragraph{Caso inductivo.} \textbf{Sin pérdida de generalidad}, suponga que $pA \overset{a}{\to} p_1A_1A_2$ y $w=auv$, entonces

$$
    (p A, \underbrace{a u v}_w) \vdash_{\mathcal{P}}^n\left(p^{\prime}, \epsilon\right) \text { ssi }(p A, a u v) \vdash_{\mathcal{P}}\left(p_1 A_1 A_2, u v\right) \vdash_{\mathcal{P}}^i\left(p_2 A_2, v\right) \vdash_{\mathcal{P}}^j\left(p^{\prime}, \epsilon\right)
$$
\begin{align*}
     & \text{ssi }  \left(p_1 A_1, u\right) \vdash_{\mathcal{P}}^i\left(p_2, \epsilon\right) \text { y } \quad\left(p_2 A_2, v\right) \vdash_{\mathcal{P}}^j\left(p^{\prime}, \epsilon\right)                                  \\
     & \text {ssi }  \left(q\left(p_1, A_1, p_2\right), u\right) \vdash_{\mathcal{P}^{\prime}}^i(q, \epsilon) \text { y } \quad\left(q\left(p_2, A_2, p^{\prime}\right), v\right) \vdash_{\mathcal{P}^{\prime}}^j(q, \epsilon) \\
     & \text {ssi }  \left.\left(q\left(p, A, p^{\prime}\right), auv\right) \vdash_{\mathcal{P}}\left(q\left(p_1, A_1, p_2\right)\left(p_2, A_2, q\right)\right), u v\right) \vdash_{\mathcal{P}}^{i+j}(q, \epsilon)
\end{align*}

\hfill $\blacksquare$

\paragraph{Paso 2.} Sea $\ca{P}=(\{q\},\Sigma,\Gamma,\Delta,q,\bot,\{q\})$ un PDA con \textbf{UN solo estado}. Contruimos la gramática:
$$
    \ca{G} = (V, \Sigma, P, \bot)
$$
\begin{itemize}
    \item $V=\gamma$.
    \item Si $qA \overset{\epsilon}{\to} q\alpha \in \Delta$ entonces $A \to \alpha \in P$
    \item Si $qA \overset{a}{\to} q\alpha \in \Delta$ entonces $A \to a\alpha \in P$
\end{itemize}
La demostración de este paso queda como ejercicio propuesto al lector.

% \section{Extracción de información}

\subsection{Extracción}

\subsection{Enumeración de resultados: Autómatas con anotaciones}

%----------------------------------------------------------------------------------------

\end{document}
