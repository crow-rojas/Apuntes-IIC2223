\section{Lenguajes libres de contexto}

\subsection{Gramáticas libres de contexto}

\subsubsection{Gramáticas}

\paragraph{Definición.} Una \textbf{gramática libre de contexto} (CFG) es una tupla:
\alignformula{
    \ca{G} = (V, \Sigma, P, S)
}
\begin{itemize}
    \item $V$ es un conjunto finito de \textbf{variables} o \textbf{no-terminales}.
    \item $\Sigma$ es el alfabeto finito (o \textbf{terminales}) tal que $\Sigma \cap V = \varnothing$.
    \item $P \subseteq V \times(V \cup \Sigma)^*$ es un subconjunto finito de \textbf{reglas} o \textbf{producciones}.
    \item $S \in V$ es la \textbf{variable inicial}.
\end{itemize}

\ejemplo{}{}{
    Considere la gramática $\ca{G} = (V, \Sigma, P, S)$ tal que:
    \begin{itemize}
        \item $V = \{X, Y\}$
        \item $\Sigma = \{a,b\}$
        \item $\{(X, aXb), (X, Y), (Y, \epsilon)\}$
        \item $S = X$
              \begin{align*}
                  \ca{G}: \quad  X & \to aXb      \\
                  X                & \to Y        \\
                  Y                & \to \epsilon
              \end{align*}
    \end{itemize}
}

\paragraph{Notación.} En este texto:
\begin{itemize}
    \item Para las \textbf{variables} en una gramática usaremos letras mayúsculas: $X,Y,Z,A,B,C,\ldots$
    \item Para los \textbf{terminales} en una gramática usaremos letras minúsculas: $a,b,c,\ldots$
    \item Para palabras en $(V \cup \Sigma)^*$ usaremos símbolos: $\alpha, \beta, \gamma, \ldots$
    \item Para una producción $(A,\alpha) \in P$ la escribimos como: $A \to \alpha$
\end{itemize}

\paragraph{Simplificación.} Si tenemos un conjunto de reglas de la forma:
$$
    \begin{array}{lll}
        X & \rightarrow & \alpha_1 \\
        X & \rightarrow & \alpha_2 \\
          & \cdots      &          \\
        X & \rightarrow & \alpha_n
    \end{array}
$$
entonces escribimos estas reglas \textbf{sucintamente} como
$$
    X \to \alpha_1 \mid \alpha_2 \mid \cdots \mid \alpha_n
$$
Recordando que $\alpha_1, \alpha_2, \ldots, \alpha_n \in (V \cup \Sigma)^*$.

\ejemplo{}{}{
    La gramática del ejemplo anterior:
    \begin{align*}
        \ca{G}: \quad  X & \to aXb      \\
        X                & \to Y        \\
        Y                & \to \epsilon
    \end{align*}
    Podemos escribirla en notación \textbf{sucinta} como:
    \begin{align*}
        \ca{G}: \quad X & \to aXb \mid Y \\
        Y               & \to \epsilon
    \end{align*}
}

\paragraph{Producciones.} Sea $\ca{G}$ una CFG. Definimos la relación $\Rightarrow \subseteq(V \cup \Sigma)^* \times(V \cup \Sigma)^*$ de \textbf{producción} tal que:
\alignformula{
    \alpha \cdot X \cdot \beta \Rightarrow \alpha \cdot \gamma \cdot \beta \quad \text { si, y solo si, } \quad(X \rightarrow \gamma) \in P
}
para todo $X \in V$ y $\alpha, \beta, \gamma \in(V \cup \Sigma)^*$. \medbreak

Si $\alpha X \beta \Rightarrow \alpha \gamma \beta$ entonces decimos que:
\begin{itemize}
    \item $\alpha X \beta$ \textbf{produce} $\alpha \gamma \beta$ o
    \item $\alpha \gamma \beta$ \textbf{es producible} desde $\alpha X \beta$.
    \item $\alpha X \beta \Rightarrow \alpha \gamma \beta$ es \textbf{reemplazar} $\gamma$ en $X$ en la palabra $\alpha X \beta$.
\end{itemize}

\paragraph{Derivaciones.} Sea $\ca{G}$ una CFG. Dadas dos palabras $\alpha, \beta \in(V \cup \Sigma)^*$ decimos que $\alpha$ \textbf{deriva} $\beta$:
\alignformula{
    \alpha \overunder{\Rightarrow}{}{*} \beta
}
si existe $\alpha_1, \alpha_2, \ldots, \alpha_n \in(V \cup \Sigma)^*$ tal que: $\alpha \Rightarrow \alpha_1 \Rightarrow \alpha_2 \Rightarrow \ldots \Rightarrow \beta$, con $\overunder{\Rightarrow}{}{*}$ la \textbf{clausura refleja y transitiva} de $\Rightarrow$, esto es:
\begin{enumerate}
    \item $\alpha \overunder{\Rightarrow}{}{*} \alpha$
    \item $\alpha \overunder{\Rightarrow}{}{*} \beta$ si, y sólo si, existe $\gamma$ tal que $\alpha \overunder{\Rightarrow}{}{*} \gamma$ y $\gamma \Rightarrow \beta$.
\end{enumerate}
para todo $\alpha, \beta \in (V \cup \Sigma)^*$. Notemos que $\Rightarrow$ y $\overunder{\Rightarrow}{}{*}$ son relaciones entre palabras en $(V \cup \Sigma)^*$.

\paragraph{Lenguaje.} Sea $\ca{G}$ una CFG. El \textbf{lenguaje} de una gramática $\ca{G}$ se define como:
\alignformula{
    \mathcal{L}(\mathcal{G})=\left\{w \in \Sigma^* \mid S \stackrel{\star}{\Rightarrow} w\right\}
}

$\ca{L}(\ca{G})$ son todas las palabras en $\Sigma^*$ que se pueden derivar desde $S$.

\ejemplo{}{}{
    Sea $\ca{G}$ una CFG tal que:
    \begin{align*}
        \ca{G}: \quad X & \to aXb \mid Y \\
        Y               & \to \epsilon
    \end{align*}
    \begin{itemize}
        \item Como $X \overunder{\Rightarrow}{}{*} aaabbb$, entonces $aaabbb \in \ca{L}(\ca{G})$.
        \item En general, uno puede demostrar por \textbf{inducción} que:
              $$
                  \mathcal{L}(\mathcal{G})=\left\{a^n b^n \mid n \geq 0\right\}
              $$
    \end{itemize}
}

\paragraph{Lenguaje libre de contexto.} Diremos que $L \subseteq \Sigma^*$ es un \textbf{lenguaje libre de contexto} si, y sólo si, existe una gramática libre de contexto $\ca{G}$ tal que:
\alignformula{
    L = \ca{L}(\ca{G})
}

\ejemplo{}{}{
    Los siguientes son lenguajes libres de contexto:
    \begin{itemize}
        \item $L=\left\{a^n b^n \mid n \geq 0\right\}$
        \item $\text{Par}=\left\{w \in\{a, b\}^* \mid w \text { tiene largo par }\right\}$
        \item $\text{Pal}=\left\{w \in\{a, b\}^* \mid w=w^{\mathrm{rev}}\right\}$
    \end{itemize}
}

\subsubsection{Árboles y derivaciones}

\subsection{Simplificación de gramáticas}
\subsection{Forma normal de Chomsky}
\subsection{Lema de bombeo para lenguajes libres de contexto}
\subsection{Algoritmo CKY}